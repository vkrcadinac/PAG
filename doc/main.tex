% generated by GAPDoc2LaTeX from XML source (Frank Luebeck)
\documentclass[a4paper,11pt]{report}

\usepackage[top=37mm,bottom=37mm,left=27mm,right=27mm]{geometry}
\sloppy
\pagestyle{myheadings}
\usepackage{amssymb}
\usepackage[utf8]{inputenc}
\usepackage{makeidx}
\makeindex
\usepackage{color}
\definecolor{FireBrick}{rgb}{0.5812,0.0074,0.0083}
\definecolor{RoyalBlue}{rgb}{0.0236,0.0894,0.6179}
\definecolor{RoyalGreen}{rgb}{0.0236,0.6179,0.0894}
\definecolor{RoyalRed}{rgb}{0.6179,0.0236,0.0894}
\definecolor{LightBlue}{rgb}{0.8544,0.9511,1.0000}
\definecolor{Black}{rgb}{0.0,0.0,0.0}

\definecolor{linkColor}{rgb}{0.0,0.0,0.554}
\definecolor{citeColor}{rgb}{0.0,0.0,0.554}
\definecolor{fileColor}{rgb}{0.0,0.0,0.554}
\definecolor{urlColor}{rgb}{0.0,0.0,0.554}
\definecolor{promptColor}{rgb}{0.0,0.0,0.589}
\definecolor{brkpromptColor}{rgb}{0.589,0.0,0.0}
\definecolor{gapinputColor}{rgb}{0.589,0.0,0.0}
\definecolor{gapoutputColor}{rgb}{0.0,0.0,0.0}

%%  for a long time these were red and blue by default,
%%  now black, but keep variables to overwrite
\definecolor{FuncColor}{rgb}{0.0,0.0,0.0}
%% strange name because of pdflatex bug:
\definecolor{Chapter }{rgb}{0.0,0.0,0.0}
\definecolor{DarkOlive}{rgb}{0.1047,0.2412,0.0064}


\usepackage{fancyvrb}

\usepackage{mathptmx,helvet}
\usepackage[T1]{fontenc}
\usepackage{textcomp}


\usepackage[
            pdftex=true,
            bookmarks=true,        
            a4paper=true,
            pdftitle={Written with GAPDoc},
            pdfcreator={LaTeX with hyperref package / GAPDoc},
            colorlinks=true,
            backref=page,
            breaklinks=true,
            linkcolor=linkColor,
            citecolor=citeColor,
            filecolor=fileColor,
            urlcolor=urlColor,
            pdfpagemode={UseNone}, 
           ]{hyperref}

\newcommand{\maintitlesize}{\fontsize{50}{55}\selectfont}

% write page numbers to a .pnr log file for online help
\newwrite\pagenrlog
\immediate\openout\pagenrlog =\jobname.pnr
\immediate\write\pagenrlog{PAGENRS := [}
\newcommand{\logpage}[1]{\protect\write\pagenrlog{#1, \thepage,}}
%% were never documented, give conflicts with some additional packages

\newcommand{\GAP}{\textsf{GAP}}

%% nicer description environments, allows long labels
\usepackage{enumitem}
\setdescription{style=nextline}

%% depth of toc
\setcounter{tocdepth}{1}





%% command for ColorPrompt style examples
\newcommand{\gapprompt}[1]{\color{promptColor}{\bfseries #1}}
\newcommand{\gapbrkprompt}[1]{\color{brkpromptColor}{\bfseries #1}}
\newcommand{\gapinput}[1]{\color{gapinputColor}{#1}}


\begin{document}

\logpage{[ 0, 0, 0 ]}
\begin{titlepage}
\mbox{}\vfill

\begin{center}{\maintitlesize \textbf{ PAG \mbox{}}}\\
\vfill

\hypersetup{pdftitle= PAG }
\markright{\scriptsize \mbox{}\hfill  PAG  \hfill\mbox{}}
{\Huge \textbf{ Prescribed Automorphism Groups \mbox{}}}\\
\vfill

{\Huge  0.2.1 \mbox{}}\\[1cm]
{ 14 April 2023 \mbox{}}\\[1cm]
\mbox{}\\[2cm]
{\Large \textbf{ Vedran Krcadinac\\
    \mbox{}}}\\
\hypersetup{pdfauthor= Vedran Krcadinac\\
    }
\end{center}\vfill

\mbox{}\\
{\mbox{}\\
\small \noindent \textbf{ Vedran Krcadinac\\
    }  Email: \href{mailto://vedran.krcadinac@math.hr} {\texttt{vedran.krcadinac@math.hr}}\\
  Homepage: \href{https://web.math.pmf.unizg.hr/~krcko/homepage.html} {\texttt{https://web.math.pmf.unizg.hr/\texttt{\symbol{126}}krcko/homepage.html}}\\
  Address: \begin{minipage}[t]{8cm}\noindent
 University of Zagreb, Faculty of Science,\\
 Department of Mathematics\\
 Bijenicka cesta 30, HR-10000 Zagreb, Croatia\\
 \end{minipage}
}\\
\end{titlepage}

\newpage\setcounter{page}{2}
{\small 
\section*{Abstract}
\logpage{[ 0, 0, 1 ]}
 \textsf{PAG} is a \textsf{GAP} package for constructing combinatorial objects with prescribed automorphism
groups. \mbox{}}\\[1cm]
{\small 
\section*{Copyright}
\logpage{[ 0, 0, 2 ]}
 \index{License} {\copyright} 2023 by Vedran Krcadinac

 The \textsf{PAG} package is free software; you can redistribute it and/or modify it under the
terms of the \href{http://www.fsf.org/licenses/gpl.html} {GNU General Public License} as published by the Free Software Foundation; either version 2 of the License,
or (at your option) any later version. \mbox{}}\\[1cm]
{\small 
\section*{Acknowledgements}
\logpage{[ 0, 0, 3 ]}
 Development of the \textsf{PAG} package has been supported by the Croatian Science Foundation under the
project IP-2020-02-9752. \mbox{}}\\[1cm]
\newpage

\def\contentsname{Contents\logpage{[ 0, 0, 4 ]}}

\tableofcontents
\newpage

  
\chapter{\textcolor{Chapter }{The PAG Package}}\label{The PAG Package}
\logpage{[ 1, 0, 0 ]}
\hyperdef{L}{X84BC1F5D7BD3CF55}{}
{
  \index{PAG} \emph{Prescribed Automorphism Groups} (\textsf{PAG}) is a \textsf{GAP} package for constructing combinatorial objects with prescribed automorphism
groups. 

 
\section{\textcolor{Chapter }{Getting Started}}\label{Getting Started}
\logpage{[ 1, 1, 0 ]}
\hyperdef{L}{X7B1863E17896BCE1}{}
{
  The package is loaded by 
\begin{Verbatim}[commandchars=!@|,fontsize=\small,frame=single,label=Example]
  !gapprompt@gap>| !gapinput@LoadPackage("PAG"); |
\end{Verbatim}
 Let us present a small example from the paper \cite{VK18}. In Theorem 8.1, a simple 5-(16,7,10) design with the following automorphism
group was constructed. 
\begin{Verbatim}[commandchars=!@|,fontsize=\small,frame=single,label=Example]
  !gapprompt@gap>| !gapinput@g:=Group((2,3,4)(5,6,7,8,9,10)(11,12,13,14,15,16), |
  !gapprompt@>| !gapinput@(1,5)(2,12)(3,15)(4,8)(6,14)(7,16)(9,10)(11,13));|
\end{Verbatim}
 The design can be obtained by typing 
\begin{Verbatim}[commandchars=@|A,fontsize=\small,frame=single,label=Example]
  @gapprompt|gap>A @gapinput|KramerMesnerSearch(5,16,7,10,g);A
  Computing t-subset orbit representatives...
  28
  Computing k-subset orbit representatives...
  71
  Computing the Kramer-Mesner matrix...
  [ 29, 72 ]
  Starting solver...
  No BOUNDS 
  The RHS is fixed !
  No upper bounds: 0/1 variables are assumed 
  
  Orthogonal defect: 26.953339
  First reduction successful
  Orthogonal defect: 20.216092
  Second reduction successful
  .
  .
  .
\end{Verbatim}
 Comments during the calculation can be supressed by setting global options. 
\begin{Verbatim}[commandchars=!@|,fontsize=\small,frame=single,label=Example]
  !gapprompt@gap>| !gapinput@PAGGlobalOptions.Silent:=true;|
  true
  !gapprompt@gap>| !gapinput@d:=KramerMesnerSearch(5,16,7,10,g);|
  [ rec( autSubgroup := Group([ (2,3,4)(5,6,7,8,9,10)(11,12,13,14,15,16), 
            (1,5)(2,12)(3,15)(4,8)(6,14)(7,16)(9,10)(11,13) ]), 
        blocks := [ [ 1, 2, 3, 4, 5, 6, 13 ], [ 1, 2, 3, 4, 5, 6, 14 ], 
            [ 1, 2, 3, 4, 5, 7, 9 ], [ 1, 2, 3, 4, 5, 7, 12 ], 
            [ 1, 2, 3, 4, 5, 9, 16 ], [ 1, 2, 3, 4, 5, 10, 12 ], 
            [ 1, 2, 3, 4, 5, 10, 13 ], [ 1, 2, 3, 4, 5, 11, 12 ], 
            [ 1, 2, 3, 4, 5, 11, 16 ], [ 1, 2, 3, 4, 5, 12, 14 ], 
            [ 1, 2, 3, 4, 6, 7, 14 ], [ 1, 2, 3, 4, 6, 7, 15 ], 
  .
  .
  .
\end{Verbatim}
 The output is a list of non-isomorphic designs in the \textsf{Design} package format  (\textbf{DESIGN: Design}). We can check that it is really a 5-design. 
\begin{Verbatim}[commandchars=!@|,fontsize=\small,frame=single,label=Example]
  !gapprompt@gap>| !gapinput@List(d,AllTDesignLambdas);|
  [ [ 2080, 910, 364, 130, 40, 10 ] ]
\end{Verbatim}
 The output is large because the \textsf{Design} format includes a list of all blocks, and 5-(16,7,10) designs have 2080
blocks. Instead, we can ask just for the base blocks. 
\begin{Verbatim}[commandchars=!@|,fontsize=\small,frame=single,label=Example]
  !gapprompt@gap>| !gapinput@bb:=KramerMesnerSearch(5,16,7,10,g,rec(BaseBlocks:=true));|
  [ [ [ 1, 2, 3, 4, 5, 6, 13 ], [ 1, 2, 3, 4, 5, 6, 14 ], 
        [ 1, 2, 3, 5, 6, 7, 11 ], [ 1, 2, 3, 5, 6, 8, 9 ], 
        [ 1, 2, 3, 5, 6, 9, 10 ], [ 1, 2, 3, 5, 6, 9, 12 ], 
        [ 1, 2, 3, 5, 6, 10, 15 ], [ 1, 2, 3, 5, 6, 14, 16 ], 
        [ 1, 2, 3, 5, 8, 11, 12 ], [ 1, 2, 5, 6, 7, 8, 16 ], 
        [ 1, 2, 5, 6, 7, 9, 14 ], [ 1, 2, 5, 6, 7, 12, 13 ], 
        [ 1, 2, 5, 6, 7, 14, 15 ] ], 
    [ [ 1, 2, 3, 4, 5, 6, 8 ], [ 1, 2, 3, 4, 5, 6, 14 ], 
        [ 1, 2, 3, 5, 6, 7, 11 ], [ 1, 2, 3, 5, 6, 9, 12 ], 
        [ 1, 2, 3, 5, 6, 10, 12 ], [ 1, 2, 3, 5, 6, 10, 16 ], 
        [ 1, 2, 3, 5, 6, 12, 13 ], [ 1, 2, 3, 5, 6, 14, 15 ], 
        [ 1, 2, 3, 5, 8, 11, 12 ], [ 1, 2, 5, 6, 7, 8, 9 ], 
        [ 1, 2, 5, 6, 7, 9, 14 ], [ 1, 2, 5, 6, 7, 12, 13 ], 
        [ 1, 2, 5, 6, 11, 14, 16 ] ] ]
\end{Verbatim}
 In this case isomorph rejection is not performed and we get two sets of base
blocks. They can be turned into designs by calling the \texttt{BlockDesign} (\textbf{DESIGN: BlockDesign}) function: \texttt{List(bb,x-{\textgreater}BlockDesign(16,x,g));}. }

 
\section{\textcolor{Chapter }{Installation}}\label{Installation}
\logpage{[ 1, 2, 0 ]}
\hyperdef{L}{X8360C04082558A12}{}
{
  The \textsf{PAG} package requires \textsf{GAP} 4.11 and the following packages: 
\begin{itemize}
\item \textsf{Images} 1.3
\item \textsf{GRAPE} 4.8
\item \textsf{Design} 1.7
\end{itemize}
 The following packages are also loaded, if available. They are needed for a
limited number of \textsf{PAG} functions. 
\begin{itemize}
\item \textsf{AssociationSchemes} 2.0
\item \textsf{DifSets} 2.3.1
\item \textsf{GUAVA} 3.15
\end{itemize}
 The current installation file for \textsf{PAG} is available at \href{https://vkrcadinac.github.io/PAG/} {\texttt{https://vkrcadinac.github.io/PAG/}}. To install \textsf{PAG}, unpack it to the \texttt{pkg} directory of your local \textsf{GAP} installation. The package uses external binaries. To compile them on UNIX-like
environments, change to the \texttt{pkg/PAG-*} directory and call 
\begin{Verbatim}[commandchars=!@|,fontsize=\small,frame=single,label=Example]
  $ ./configure.sh
\end{Verbatim}
 This produces a \texttt{Makefile} in the current directory. Now call 
\begin{Verbatim}[commandchars=!@|,fontsize=\small,frame=single,label=Example]
  $ make all
\end{Verbatim}
 to compile the binares. They are placed in the \texttt{bin} subdirectory. Documentation in the \texttt{doc} subdirectory is already compiled and can be read in PDF, html or from within \textsf{GAP}. To recompile the documentation, call \textsf{GAP} with the \texttt{makedoc.g} file. }

 
\section{\textcolor{Chapter }{Examples: Designs}}\label{Examples: Designs}
\logpage{[ 1, 3, 0 ]}
\hyperdef{L}{X87AF9F4C789605D1}{}
{
  The \textsf{PAG} function \texttt{KramerMesnerSearch} performs a search for $t$-designs with given parameters and a given permutation group as group of
automorphisms. See the paper by B.{\nobreakspace}Schmalz{\nobreakspace}\cite{BS93} for an introduction to the Kramer-Mesner approach to constructing $t$-designs. Our first two examples are from this paper. The original paper of
Earl Kramer and Dale Mesner is{\nobreakspace}\cite{KM76}. 
\subsection{\textcolor{Chapter }{6-(14,7,4) Designs}}\label{6-(14,7,4) Designs}
\logpage{[ 1, 3, 1 ]}
\hyperdef{L}{X85DE43FE83E85401}{}
{
  The summary about known 6-designs on page 130 of \cite{BS93} mentions that there are exactly two 6-(14,7,4) designs with cyclic derived
designs. This means that the two 6-designs have automorphisms of order 13.
They can be constructed by the following \textsf{GAP} commands. 
\begin{Verbatim}[commandchars=!@|,fontsize=\small,frame=single,label=Example]
  !gapprompt@gap>| !gapinput@g:=Group(CyclicPerm(13));|
  Group([ (1,2,3,4,5,6,7,8,9,10,11,12,13) ])
  !gapprompt@gap>| !gapinput@d:=KramerMesnerSearch(6,14,7,4,g);;|
  !gapprompt@gap>| !gapinput@List(d,AllTDesignLambdas);|
  [ [ 1716, 858, 396, 165, 60, 18, 4 ], [ 1716, 858, 396, 165, 60, 18, 4 ] ]
\end{Verbatim}
 The solver quickly finds 24 solutions of the Kramer-Mesner system. Most of the
computation time is used to eliminate isomorphic designs. This can be turned
off: 
\begin{Verbatim}[commandchars=!@|,fontsize=\small,frame=single,label=Example]
  !gapprompt@gap>| !gapinput@d2:=KramerMesnerSearch(6,14,7,4,g,rec(NonIsomorphic:=false));;|
  !gapprompt@gap>| !gapinput@Size(d2);|
  30
  !gapprompt@gap>| !gapinput@Size(AsSet(d2));|
  24
\end{Verbatim}
 Now we get a list of 30 designs. By default, A.{\nobreakspace}Wassermann's LLL
solver \cite{AW98} is used; it may return the same solution more than once. The number of
distinct designs is 24. The two non-isomorphic designs have ${\ensuremath{\mathbb Z}}_{13}$ as their full automorphism group. 
\begin{Verbatim}[commandchars=!@|,fontsize=\small,frame=single,label=Example]
  !gapprompt@gap>| !gapinput@List(d,BlockDesignAut);|
  [ Group([ (1,2,3,4,5,6,7,8,9,10,11,12,13) ]), 
    Group([ (1,2,3,4,5,6,7,8,9,10,11,12,13) ]) ]
\end{Verbatim}
 }

 
\subsection{\textcolor{Chapter }{6-(28,8,$\lambda$) Designs}}\label{6-(28,8,lambda) Designs}
\logpage{[ 1, 3, 2 ]}
\hyperdef{L}{X7835113382A11FA1}{}
{
  In \cite{BS93}, the existence of 6-(28,8,$\lambda$) designs was established for $\lambda=42$, 63, 84, and 105. The exact numbers of these designs with automorphism group $P\Gamma L(2,27)$ were computed. While the projective general linear groups are readily
available in \textsf{GAP} through the \texttt{PGL} command, there seems to be no equivalent command for semilinear groups. We can
get $P\Gamma L(2,27)$ using the \textsf{FinInG} package, as the collineation group of the projective line over $GF(27)$. 
\begin{Verbatim}[commandchars=!@|,fontsize=\small,frame=single,label=Example]
  !gapprompt@gap>| !gapinput@LoadPackage("FinInG");|
  !gapprompt@gap>| !gapinput@g1:=CollineationGroup(ProjectiveSpace(1,27));|
  The FinInG collineation group PGammaL(2,27)
\end{Verbatim}
 We need a permutation representation of this group on 28 points. 
\begin{Verbatim}[commandchars=!@|,fontsize=\small,frame=single,label=Example]
  !gapprompt@gap>| !gapinput@g:=Image(ActionOnAllProjPoints(g1));|
  Group([ (3,28,27,26,25,24,23,22,21,20,19,18,17,4,16,15,14,13,12,11,10,9,8,7,6,5), 
    (1,2,4)(5,8,24)(6,21,10)(7,16,15)(9,25,28)(11,13,14)(12,27,23)(17,26,18)
    (19,20,22), (5,7,13)(6,10,21)(8,16,14)(9,18,22)(11,24,15)(12,27,23)(17,19,25)
    (20,28,26) ])
\end{Verbatim}
 Alternatively, we can get $P\Gamma L(2,27)$ from the library of small primitive permutation groups. 
\begin{Verbatim}[commandchars=!@|,fontsize=\small,frame=single,label=Example]
  !gapprompt@gap>| !gapinput@PrimitiveGroupsOfDegree(28);|
  [ PGL(2, 7), PSL(2, 8), PGammaL(2, 8), PSU(3, 3), PGammaU(3, 3), PSp(6, 2), A(8), 
    S(8), PSL(2, 27), PGL(2, 27), PSL(2, 27):3, PGammaL(2, 27), A(28), S(28) ]
\end{Verbatim}
 Now we can construct the designs with $\lambda=42$. 
\begin{Verbatim}[commandchars=!@|,fontsize=\small,frame=single,label=Example]
  !gapprompt@gap>| !gapinput@d:=KramerMesnerSearch(6,28,8,42,g,rec(BaseBlocks:=true));;|
  !gapprompt@gap>| !gapinput@Size(AsSet(d));|
  3
\end{Verbatim}
 Most of the CPU time in the example above was used to compute the
Kramer-Mesner matrix. The left side of the Kramer-Mesner system is the same
matrix for all $\lambda$, so we can compute it once and reuse it to save time. 
\begin{Verbatim}[commandchars=!@|,fontsize=\small,frame=single,label=Example]
  !gapprompt@gap>| !gapinput@tsub:=SubsetOrbitRep(g,28,6);;|
  !gapprompt@gap>| !gapinput@ksub:=SubsetOrbitRep(g,28,8);;|
  !gapprompt@gap>| !gapinput@m:=KramerMesnerMat(g,tsub,ksub);;|
\end{Verbatim}
 Now we can quickly get the exact numbers of designs from the paper \cite{BS93}. 
\begin{Verbatim}[commandchars=!@|,fontsize=\small,frame=single,label=Example]
  !gapprompt@gap>| !gapinput@Size(AsSet(SolveKramerMesner(ExpandMatRHS(m,42))));|
  3
  !gapprompt@gap>| !gapinput@Size(AsSet(SolveKramerMesner(ExpandMatRHS(m,63))));|
  367
  !gapprompt@gap>| !gapinput@Size(AsSet(SolveKramerMesner(ExpandMatRHS(m,84))));|
  21743
  !gapprompt@gap>| !gapinput@Size(AsSet(SolveKramerMesner(ExpandMatRHS(m,105))));|
  38277
\end{Verbatim}
 }

 
\subsection{\textcolor{Chapter }{2-(81,6,2) Designs}}\label{2-(81,6,2) Designs}
\logpage{[ 1, 3, 3 ]}
\hyperdef{L}{X82F01F238596D36C}{}
{
  The first simple 2-(81,6,2) design was recently found by
A.{\nobreakspace}Nakic{\nobreakspace}\cite{AN21}. Here are the base blocks of this design copy-pasted from the paper. 
\begin{Verbatim}[commandchars=!@|,fontsize=\small,frame=single,label=Example]
  !gapprompt@gap>| !gapinput@bb:=[[[0,0,0,0],[0,0,0,1],[0,0,0,2],[0,1,0,0],[0,1,0,1],[0,1,0,2]],|
  !gapprompt@>| !gapinput@[[0,0,0,0],[0,0,1,1],[0,0,2,2],[2,1,0,0],[2,1,1,1],[2,1,2,2]],|
  !gapprompt@>| !gapinput@[[0,0,0,0],[0,1,1,1],[0,2,2,2],[0,0,1,0],[0,1,2,1],[0,2,0,2]],|
  !gapprompt@>| !gapinput@[[0,0,0,0],[0,1,2,0],[0,2,1,0],[2,0,2,1],[2,1,1,1],[2,2,0,1]],|
  !gapprompt@>| !gapinput@[[0,0,0,0],[1,0,0,0],[2,0,0,0],[0,2,2,1],[1,2,2,1],[2,2,2,1]],|
  !gapprompt@>| !gapinput@[[0,0,0,0],[1,0,1,0],[2,0,2,0],[0,1,0,0],[1,1,1,0],[2,1,2,0]],|
  !gapprompt@>| !gapinput@[[0,0,0,0],[1,0,1,1],[2,0,2,2],[0,0,2,0],[1,0,0,1],[2,0,1,2]],|
  !gapprompt@>| !gapinput@[[0,0,0,0],[1,0,2,0],[2,0,1,0],[0,2,1,1],[1,2,0,1],[2,2,2,1]],|
  !gapprompt@>| !gapinput@[[0,0,0,0],[1,0,2,2],[2,0,1,1],[0,1,2,1],[1,1,1,0],[2,1,0,2]],|
  !gapprompt@>| !gapinput@[[0,0,0,0],[1,1,0,0],[2,2,0,0],[0,2,0,1],[1,0,0,1],[2,1,0,1]],|
  !gapprompt@>| !gapinput@[[0,0,0,0],[1,1,0,1],[2,2,0,2],[0,2,2,0],[1,0,2,1],[2,1,2,2]],|
  !gapprompt@>| !gapinput@[[0,0,0,0],[1,1,2,0],[2,2,1,0],[0,0,2,1],[1,1,1,1],[2,2,0,1]],|
  !gapprompt@>| !gapinput@[[0,0,0,0],[1,1,2,1],[2,2,1,2],[0,2,1,1],[1,0,0,2],[2,1,2,0]],|
  !gapprompt@>| !gapinput@[[0,0,0,0],[1,1,2,2],[2,2,1,1],[0,2,2,0],[1,0,1,2],[2,1,0,1]],|
  !gapprompt@>| !gapinput@[[0,0,0,0],[1,2,1,2],[2,1,2,1],[0,0,2,1],[1,2,0,0],[2,1,1,2]],|
  !gapprompt@>| !gapinput@[[0,0,0,0],[1,2,2,0],[2,1,1,0],[0,2,2,1],[1,1,1,1],[2,0,0,1]]]*Z(3)^0;;|
\end{Verbatim}
 The points of this design are elements of the 4-dimensional vector space $V$ over $GF(3)$. Here is how to get the desing in the \textsf{Design} package format. 
\begin{Verbatim}[commandchars=!@|,fontsize=\small,frame=single,label=Example]
  !gapprompt@gap>| !gapinput@V:=Tuples([0,1,2],4)*Z(3)^0;;|
  !gapprompt@gap>| !gapinput@d1:=Union(List(bb,y->List(V,x->AsSet(x+y))));;|
  !gapprompt@gap>| !gapinput@d:=BlockDesign(81,List(d1,y->List(y,x->Position(V,x))));;|
  !gapprompt@gap>| !gapinput@AllTDesignLambdas(d);|
  [ 432, 32, 2 ]
\end{Verbatim}
 The full automorphism group of the design is of order 2592. It is a semidirect
product of the additive group of $V$ and a group of order 32. 
\begin{Verbatim}[commandchars=!@|,fontsize=\small,frame=single,label=Example]
  !gapprompt@gap>| !gapinput@aut:=BlockDesignAut(d);|
  <permutation group with 5 generators>
  !gapprompt@gap>| !gapinput@Size(aut);|
  2592
  !gapprompt@gap>| !gapinput@StructureDescription(aut);|
  "(C3 x C3 x C3 x C3) : (C16 : C2)"
\end{Verbatim}
 This group has three subgroups of order 648 up to conjugation. We can use the
second subgroup to construct four more simple 2-(81,6,2) designs. 
\begin{Verbatim}[commandchars=!@|,fontsize=\small,frame=single,label=Example]
  !gapprompt@gap>| !gapinput@g:=Filtered(AllSubgroupsConjugation(aut),x->Size(x)=648);|
  [ <permutation group of size 648 with 7 generators>, 
    <permutation group of size 648 with 7 generators>, 
    <permutation group of size 648 with 7 generators> ]
  !gapprompt@gap>| !gapinput@dd:=KramerMesnerSearch(2,81,6,2,g[2]);;|
  !gapprompt@gap>| !gapinput@List(dd,x->Size(AutomorphismGroup(x)));|
  [ 1296, 2592, 3888, 1944, 15552 ]
\end{Verbatim}
 Two of the new designs have larger full automorphism groups than the design
from{\nobreakspace}\cite{AN21}. Using their subgroups, more simple 2-(81,6,2) designs can be constructed. }

 
\subsection{\textcolor{Chapter }{Quasi-symmetric 2-(56,16,18) Designs}}\label{Quasi-symmetric 2-(56,16,18) Designs}
\logpage{[ 1, 3, 4 ]}
\hyperdef{L}{X7C9ADE74821A9A39}{}
{
  Here is how the quasi-symmetric 2-(56,16,18) designs with intersection numbers $x=4$, $y=8$ from the paper{\nobreakspace}\cite{KV16} can be constructed. 
\begin{Verbatim}[commandchars=!@|,fontsize=\small,frame=single,label=Example]
  !gapprompt@gap>| !gapinput@g:=Group((1,2,3,4,5)(6,7,8,9,10)(11,12,13,14,15)(16,17,18,19,20)|
  !gapprompt@>| !gapinput@  (21,22,23,24,25)(26,27,28,29,30)(31,32,33,34,35)(36,37,38,39,40)|
  !gapprompt@>| !gapinput@  (41,42,43,44,45)(46,47,48,49,50)(51,52,53,54,55),|
  !gapprompt@>| !gapinput@(1,6,8)(2,21,26)(3,32,34)(4,11,5)(7,15,22)(9,16,13)(10,29,17)|
  !gapprompt@>| !gapinput@  (12,33,30)(14,19,31)(18,23,35)(24,28,36)(25,37,39)(27,38,40)|
  !gapprompt@>| !gapinput@  (42,51,49)(43,52,45)(44,46,47)(48,54,53)(50,56,55));|
  <permutation group with 2 generators>
  !gapprompt@gap>| !gapinput@d:=KramerMesnerSearch(2,56,16,18,g,rec(IntersectionNumbers:=[4,8]));;|
  !gapprompt@gap>| !gapinput@Size(d);|
  3
\end{Verbatim}
 We check that they have all required properties and compute their full
automorphism groups: 
\begin{Verbatim}[commandchars=!@|,fontsize=\small,frame=single,label=Example]
  !gapprompt@gap>| !gapinput@List(d,AllTDesignLambdas);|
  [ [ 231, 66, 18 ], [ 231, 66, 18 ], [ 231, 66, 18 ] ]
  !gapprompt@gap>| !gapinput@List(d,IntersectionNumbers);|
  [ [ 4, 8 ], [ 4, 8 ], [ 4, 8 ] ]
  !gapprompt@gap>| !gapinput@aut:=List(d,BlockDesignAut);;|
  !gapprompt@gap>| !gapinput@List(aut,StructureDescription);|
  [ "(C2 x C2 x C2 x C2) : S5", "(C2 x C2 x C2 x C2) : A5", "PSL(3,4) : C2" ]
\end{Verbatim}
 }

 }

 
\section{\textcolor{Chapter }{Examples: Latin Squares}}\label{Examples: Latin Squares}
\logpage{[ 1, 4, 0 ]}
\hyperdef{L}{X7CCA2066781B6D56}{}
{
  See{\nobreakspace}\cite{KD15} for an introduction to Latin squares and definitions of isotopy, paratopy,
etc. Multiplication tables of groups are examples of Latin squares. 
\begin{Verbatim}[commandchars=!@|,fontsize=\small,frame=single,label=Example]
  !gapprompt@gap>| !gapinput@CayleyTableOfGroup(CyclicGroup(7));|
  [ [ 1, 2, 3, 4, 5, 6, 7 ], 
    [ 2, 3, 4, 5, 6, 7, 1 ], 
    [ 3, 4, 5, 6, 7, 1, 2 ], 
    [ 4, 5, 6, 7, 1, 2, 3 ], 
    [ 5, 6, 7, 1, 2, 3, 4 ], 
    [ 6, 7, 1, 2, 3, 4, 5 ], 
    [ 7, 1, 2, 3, 4, 5, 6 ] ]
\end{Verbatim}
 We can construct more examples by prescribing symmetry groups. The \textsf{PAG} function \texttt{KramerMesnerMOLS} performs a search for sets of $s$ mutually orthogonal Latin squares (MOLS) of order $n$ and a given permutation group as autotopy or autoparatopy group. The group
must act on the $s+2$ point classes of the corresponding transversal design. By{\nobreakspace}\cite{RF12} and{\nobreakspace}\cite{SVW12}, an autotopism of order 5 of a Latin square of order 7 must have the
following cycle structure. 
\begin{Verbatim}[commandchars=!@|,fontsize=\small,frame=single,label=Example]
  !gapprompt@gap>| !gapinput@a:=MultiPerm(CyclicPerm(5),[1..7],3);|
  (1,2,3,4,5)(8,9,10,11,12)(15,16,17,18,19)
\end{Verbatim}
 There are two main classes of such Latin squares. They are multiplication
tables of non-associative quasigroups. 
\begin{Verbatim}[commandchars=!@|,fontsize=\small,frame=single,label=Example]
  !gapprompt@gap>| !gapinput@KramerMesnerMOLS(7,1,Group(a));|
  [ [ [ [ 1, 3, 2, 6, 7, 4, 5 ], 
            [ 7, 2, 4, 3, 6, 5, 1 ], 
            [ 6, 7, 3, 5, 4, 1, 2 ], 
            [ 5, 6, 7, 4, 1, 2, 3 ], 
            [ 2, 1, 6, 7, 5, 3, 4 ], 
            [ 3, 4, 5, 1, 2, 6, 7 ], 
            [ 4, 5, 1, 2, 3, 7, 6 ] ] ], 
    [ [ [ 1, 3, 5, 6, 7, 2, 4 ], 
            [ 7, 2, 4, 1, 6, 3, 5 ], 
            [ 6, 7, 3, 5, 2, 4, 1 ], 
            [ 3, 6, 7, 4, 1, 5, 2 ], 
            [ 2, 4, 6, 7, 5, 1, 3 ], 
            [ 4, 5, 1, 2, 3, 6, 7 ], 
            [ 5, 1, 2, 3, 4, 7, 6 ] ] ] ]
\end{Verbatim}
 Single Latin squares are treated as MOLS sets of size $s=1$, hence the excess brackets. When the order $n$ is a prime power, complete sets of $s=n-1$ MOLS are easily constructed from finite fields. 
\begin{Verbatim}[commandchars=!@|,fontsize=\small,frame=single,label=Example]
  !gapprompt@gap>| !gapinput@ls4:=FieldToMOLS(GF(4));|
  [ [ [ 1, 2, 3, 4 ], 
        [ 2, 1, 4, 3 ], 
        [ 3, 4, 1, 2 ], 
        [ 4, 3, 2, 1 ] ], 
    [ [ 1, 2, 3, 4 ], 
        [ 3, 4, 1, 2 ], 
        [ 4, 3, 2, 1 ], 
        [ 2, 1, 4, 3 ] ], 
    [ [ 1, 2, 3, 4 ], 
        [ 4, 3, 2, 1 ], 
        [ 2, 1, 4, 3 ], 
        [ 3, 4, 1, 2 ] ] ]
  !gapprompt@gap>| !gapinput@AreMOLS(ls4);|
  true
\end{Verbatim}
 The package \textsf{Guava} contains a function \texttt{AreMOLS} (\textbf{GUAVA: AreMOLS}) to test sets of MOLS. A famous problem is to find MOLS of order 10. The
Handbook of Combinatorial Designs{\nobreakspace}\cite{CD07}, III.5.6 contains an example of a 1-diagonally cyclic self-orthogonal Latin
square $L$ of order 10. Self-orthogonal means that $L$ is orthogonal to its transpose. In other words, the MOLS set $\{L,L^t\}$ is invariant under the following conjugation, simultaneously exchanging
rows{\textendash}columns and the two Latin squares. 
\begin{Verbatim}[commandchars=!@|,fontsize=\small,frame=single,label=Example]
  !gapprompt@gap>| !gapinput@c:=Sortex(Concatenation([11..20],[1..10],[31..40],[21..30]));|
  (1,11)(2,12)(3,13)(4,14)(5,15)(6,16)(7,17)(8,18)(9,19)(10,20)(21,
  31)(22,32)(23,33)(24,34)(25,35)(26,36)(27,37)(28,38)(29,39)(30,40)
\end{Verbatim}
 Furthermore, the example from{\nobreakspace}\cite{CD07} has an autotopism of order 9. 
\begin{Verbatim}[commandchars=!@|,fontsize=\small,frame=single,label=Example]
  !gapprompt@gap>| !gapinput@a:=MultiPerm(CyclicPerm(9),[1..10],4);|
  (1,2,3,4,5,6,7,8,9)(11,12,13,14,15,16,17,18,19)(21,22,23,24,25,26,
  27,28,29)(31,32,33,34,35,36,37,38,39)
\end{Verbatim}
 The permutations $a$ and $c$ generate an autoparatopy group of order 18 we can use to construct the
example. 
\begin{Verbatim}[commandchars=!@|,fontsize=\small,frame=single,label=Example]
  !gapprompt@gap>| !gapinput@g:=Group(a,c);;|
  !gapprompt@gap>| !gapinput@Size(g);|
  18
  !gapprompt@gap>| !gapinput@ls10:=KramerMesnerMOLS(10,2,g);;|
  !gapprompt@gap>| !gapinput@List(ls10,AreMOLS);|
  [ true, true, true, true, true ]
\end{Verbatim}
 We see that there are 5 inequivalent pairs of MOLS with $g$ as autoparatopy group. Here is one pair. 
\begin{Verbatim}[commandchars=!@|,fontsize=\small,frame=single,label=Example]
  !gapprompt@gap>| !gapinput@ls10[1];|
  [ [ [ 1, 3, 6, 9, 2, 10, 5, 7, 4, 8 ], 
        [ 5, 2, 4, 7, 1, 3, 10, 6, 8, 9 ], 
        [ 9, 6, 3, 5, 8, 2, 4, 10, 7, 1 ], 
        [ 8, 1, 7, 4, 6, 9, 3, 5, 10, 2 ], 
        [ 10, 9, 2, 8, 5, 7, 1, 4, 6, 3 ], 
        [ 7, 10, 1, 3, 9, 6, 8, 2, 5, 4 ], 
        [ 6, 8, 10, 2, 4, 1, 7, 9, 3, 5 ], 
        [ 4, 7, 9, 10, 3, 5, 2, 8, 1, 6 ], 
        [ 2, 5, 8, 1, 10, 4, 6, 3, 9, 7 ], 
        [ 3, 4, 5, 6, 7, 8, 9, 1, 2, 10 ] ], 
    [ [ 1, 5, 9, 8, 10, 7, 6, 4, 2, 3 ], 
        [ 3, 2, 6, 1, 9, 10, 8, 7, 5, 4 ], 
        [ 6, 4, 3, 7, 2, 1, 10, 9, 8, 5 ], 
        [ 9, 7, 5, 4, 8, 3, 2, 10, 1, 6 ], 
        [ 2, 1, 8, 6, 5, 9, 4, 3, 10, 7 ], 
        [ 10, 3, 2, 9, 7, 6, 1, 5, 4, 8 ], 
        [ 5, 10, 4, 3, 1, 8, 7, 2, 6, 9 ], 
        [ 7, 6, 10, 5, 4, 2, 9, 8, 3, 1 ], 
        [ 4, 8, 7, 10, 6, 5, 3, 1, 9, 2 ], 
        [ 8, 9, 1, 2, 3, 4, 5, 6, 7, 10 ] ] ]
\end{Verbatim}
 }

 
\section{\textcolor{Chapter }{Examples: Cubes of Symmetric Designs}}\label{Examples: Cubes of Symmetric Designs}
\logpage{[ 1, 5, 0 ]}
\hyperdef{L}{X7B532D6B81CC1FC2}{}
{
  Cubes of symmetric designs are studied in the paper \cite{KPT23}. Here is an example. 
\begin{Verbatim}[commandchars=!@|,fontsize=\small,frame=single,label=Example]
  !gapprompt@gap>| !gapinput@c:=DifferenceCube(Group((1,2,3,4,5,6,7)),[1,2,4],3);|
  [ [ [ 1, 1, 0, 1, 0, 0, 0 ], 
       [ 1, 0, 1, 0, 0, 0, 1 ], 
       [ 0, 1, 0, 0, 0, 1, 1 ], 
       [ 1, 0, 0, 0, 1, 1, 0 ], 
       [ 0, 0, 0, 1, 1, 0, 1 ], 
       [ 0, 0, 1, 1, 0, 1, 0 ], 
       [ 0, 1, 1, 0, 1, 0, 0 ] ], 
    [ [ 1, 0, 1, 0, 0, 0, 1 ], 
       [ 0, 1, 0, 0, 0, 1, 1 ], 
       [ 1, 0, 0, 0, 1, 1, 0 ], 
       [ 0, 0, 0, 1, 1, 0, 1 ], 
       [ 0, 0, 1, 1, 0, 1, 0 ], 
       [ 0, 1, 1, 0, 1, 0, 0 ], 
       [ 1, 1, 0, 1, 0, 0, 0 ] ], 
    [ [ 0, 1, 0, 0, 0, 1, 1 ], 
        [ 1, 0, 0, 0, 1, 1, 0 ], 
        [ 0, 0, 0, 1, 1, 0, 1 ], 
        [ 0, 0, 1, 1, 0, 1, 0 ], 
        [ 0, 1, 1, 0, 1, 0, 0 ], 
        [ 1, 1, 0, 1, 0, 0, 0 ], 
        [ 1, 0, 1, 0, 0, 0, 1 ] ], 
    [ [ 1, 0, 0, 0, 1, 1, 0 ], 
        [ 0, 0, 0, 1, 1, 0, 1 ], 
        [ 0, 0, 1, 1, 0, 1, 0 ], 
        [ 0, 1, 1, 0, 1, 0, 0 ], 
        [ 1, 1, 0, 1, 0, 0, 0 ], 
        [ 1, 0, 1, 0, 0, 0, 1 ], 
        [ 0, 1, 0, 0, 0, 1, 1 ] ], 
    [ [ 0, 0, 0, 1, 1, 0, 1 ], 
        [ 0, 0, 1, 1, 0, 1, 0 ], 
        [ 0, 1, 1, 0, 1, 0, 0 ], 
        [ 1, 1, 0, 1, 0, 0, 0 ], 
        [ 1, 0, 1, 0, 0, 0, 1 ], 
        [ 0, 1, 0, 0, 0, 1, 1 ], 
        [ 1, 0, 0, 0, 1, 1, 0 ] ], 
    [ [ 0, 0, 1, 1, 0, 1, 0 ], 
        [ 0, 1, 1, 0, 1, 0, 0 ], 
        [ 1, 1, 0, 1, 0, 0, 0 ], 
        [ 1, 0, 1, 0, 0, 0, 1 ], 
        [ 0, 1, 0, 0, 0, 1, 1 ], 
        [ 1, 0, 0, 0, 1, 1, 0 ], 
        [ 0, 0, 0, 1, 1, 0, 1 ] ], 
    [ [ 0, 1, 1, 0, 1, 0, 0 ], 
        [ 1, 1, 0, 1, 0, 0, 0 ], 
        [ 1, 0, 1, 0, 0, 0, 1 ], 
        [ 0, 1, 0, 0, 0, 1, 1 ], 
        [ 1, 0, 0, 0, 1, 1, 0 ], 
        [ 0, 0, 0, 1, 1, 0, 1 ], 
        [ 0, 0, 1, 1, 0, 1, 0 ] ] ]
\end{Verbatim}
 This is a $3$-dimensional array of zeros and ones such that all $2$-dimensional slices are incidence matrices of $(7,3,1)$ designs. For example, here is a slice obtained by varying coordinates $1,3$ and setting coordinate $2$ to $7$. 
\begin{Verbatim}[commandchars=!@|,fontsize=\small,frame=single,label=Example]
  !gapprompt@gap>| !gapinput@m:=CubeSlice(c,1,3,[7]);|
  [ [ 0, 1, 1, 0, 1, 0, 0 ], 
    [ 1, 1, 0, 1, 0, 0, 0 ], 
    [ 1, 0, 1, 0, 0, 0, 1 ], 
    [ 0, 1, 0, 0, 0, 1, 1 ], 
    [ 1, 0, 0, 0, 1, 1, 0 ], 
    [ 0, 0, 0, 1, 1, 0, 1 ], 
    [ 0, 0, 1, 1, 0, 1, 0 ] ]
  !gapprompt@gap>| !gapinput@m*TransposedMat(m);|
  [ [ 3, 1, 1, 1, 1, 1, 1 ], 
    [ 1, 3, 1, 1, 1, 1, 1 ], 
    [ 1, 1, 3, 1, 1, 1, 1 ], 
    [ 1, 1, 1, 3, 1, 1, 1 ], 
    [ 1, 1, 1, 1, 3, 1, 1 ], 
    [ 1, 1, 1, 1, 1, 3, 1 ], 
    [ 1, 1, 1, 1, 1, 1, 3 ] ]
\end{Verbatim}
 A cube of arbitrary dimension $n\ge 2$ can be constructed from a difference set in a group by calling \texttt{DifferenceCube} (\ref{DifferenceCube}). The function uses the representation of difference sets from the \textsf{DifSets} package  (\textbf{DifSets: Difference Sets}). For $n=2$, the difference cube is simply an incidence matrix of the associated
symmetric design, i.e. the development of the difference set. 
\begin{Verbatim}[commandchars=!@|,fontsize=\small,frame=single,label=Example]
  !gapprompt@gap>| !gapinput@g:=SmallGroup(15,1);|
  <pc group of size 15 with 2 generators>
  !gapprompt@gap>| !gapinput@StructureDescription(g);|
  "C15"
  !gapprompt@gap>| !gapinput@ds:=DifferenceSets(g);|
  [ [ 1, 2, 3, 4, 8, 11, 12 ] ]
  !gapprompt@gap>| !gapinput@m:=DifferenceCube(g,ds[1],2);|
  [ [ 1, 1, 1, 1, 0, 0, 0, 1, 0, 0, 1, 1, 0, 0, 0 ], 
    [ 1, 1, 0, 1, 0, 1, 1, 0, 1, 0, 0, 0, 0, 0, 1 ], 
    [ 1, 0, 0, 0, 1, 0, 0, 1, 1, 0, 0, 1, 0, 1, 1 ], 
    [ 1, 1, 0, 1, 1, 0, 0, 0, 0, 1, 0, 0, 1, 1, 0 ], 
    [ 0, 0, 1, 1, 0, 1, 0, 0, 0, 0, 0, 1, 1, 1, 1 ], 
    [ 0, 1, 0, 0, 1, 1, 0, 0, 1, 0, 1, 1, 1, 0, 0 ], 
    [ 0, 1, 0, 0, 0, 0, 1, 0, 0, 1, 1, 1, 0, 1, 1 ], 
    [ 1, 0, 1, 0, 0, 0, 0, 0, 1, 1, 1, 0, 1, 0, 1 ], 
    [ 0, 1, 1, 0, 0, 1, 0, 1, 1, 1, 0, 0, 0, 1, 0 ], 
    [ 0, 0, 0, 1, 0, 0, 1, 1, 1, 0, 1, 0, 1, 1, 0 ], 
    [ 1, 0, 0, 0, 0, 1, 1, 1, 0, 1, 0, 1, 1, 0, 0 ], 
    [ 1, 0, 1, 0, 1, 1, 1, 0, 0, 0, 1, 0, 0, 1, 0 ], 
    [ 0, 0, 0, 1, 1, 1, 0, 1, 0, 1, 1, 0, 0, 0, 1 ], 
    [ 0, 0, 1, 1, 1, 0, 1, 0, 1, 1, 0, 1, 0, 0, 0 ], 
    [ 0, 1, 1, 0, 1, 0, 1, 1, 0, 0, 0, 0, 1, 0, 1 ] ]
  !gapprompt@gap>| !gapinput@d:=BlockDesign(15,List(m,x->Positions(x,1)));;|
  !gapprompt@gap>| !gapinput@AllTDesignLambdas(d);|
  [ 15, 7, 3 ]
\end{Verbatim}
 The function \texttt{DifferenceSets} (\textbf{DifSets: DifferenceSets}) returns a list of all difference sets up to equivalence in a given group. Here
is a small $4$-dimensional $(3,2,1)$ cube. 
\begin{Verbatim}[commandchars=!@|,fontsize=\small,frame=single,label=Example]
  !gapprompt@gap>| !gapinput@c:=DifferenceCube(Group((1,2,3)),[1,2],4);|
  [ [ [ [ 1, 1, 0 ], [ 1, 0, 1 ], [ 0, 1, 1 ] ], 
        [ [ 1, 0, 1 ], [ 0, 1, 1 ], [ 1, 1, 0 ] ], 
        [ [ 0, 1, 1 ], [ 1, 1, 0 ], [ 1, 0, 1 ] ] ], 
    [ [ [ 1, 0, 1 ], [ 0, 1, 1 ], [ 1, 1, 0 ] ], 
        [ [ 0, 1, 1 ], [ 1, 1, 0 ], [ 1, 0, 1 ] ], 
        [ [ 1, 1, 0 ], [ 1, 0, 1 ], [ 0, 1, 1 ] ] ], 
    [ [ [ 0, 1, 1 ], [ 1, 1, 0 ], [ 1, 0, 1 ] ], 
        [ [ 1, 1, 0 ], [ 1, 0, 1 ], [ 0, 1, 1 ] ], 
        [ [ 1, 0, 1 ], [ 0, 1, 1 ], [ 1, 1, 0 ] ] ] ]
  !gapprompt@gap>| !gapinput@CubeTest(c);|
  [ [ 3, 2, 1 ] ]
\end{Verbatim}
 The function \texttt{CubeTest} (\ref{CubeTest}) looks at all possible slices and checks if they are incidence matrices of $(v,k,\lambda)$ designs. In the next example we construct all $3$-dimensional difference cubes of order $21$. 
\begin{Verbatim}[commandchars=!@|,fontsize=\small,frame=single,label=Example]
  !gapprompt@gap>| !gapinput@g:=AllSmallGroups(21);;|
  !gapprompt@gap>| !gapinput@List(g,StructureDescription);|
  [ "C7 : C3", "C21" ]
  !gapprompt@gap>| !gapinput@ds:=List(g,DifferenceSets);|
  [ [ [ 1, 2, 3, 9, 10 ] ], [ [ 1, 2, 7, 10, 16 ] ] ]
  !gapprompt@gap>| !gapinput@c1:=DifferenceCube(g[1],ds[1][1],3);;|
  !gapprompt@gap>| !gapinput@c2:=DifferenceCube(g[2],ds[2][1],3);;|
  !gapprompt@gap>| !gapinput@List([c1,c2],CubeTest);|
  [ [ [ 21, 5, 1 ] ], [ [ 21, 5, 1 ] ] ]
  !gapprompt@gap>| !gapinput@Size(CubeAut(c1));|
  1323
  !gapprompt@gap>| !gapinput@Size(CubeAut(c2));|
  2646
\end{Verbatim}
 The function \texttt{CubeAut} (\ref{CubeAut}) computes the full autotopy group of a cube. By setting options, full
autoparatopy groups can also be obtained. We can make a non-difference cube by
the "group cube" construction of Theorem 4.1 from \cite{KPT23}. First we search for all $(21,5,1)$ designs with blocks being difference sets in the Frobenius group of order $21$. 
\begin{Verbatim}[commandchars=!@|,fontsize=\small,frame=single,label=Example]
  !gapprompt@gap>| !gapinput@allds:=Filtered(Combinations([1..21],5),x->IsDifferenceSet(g[1],x));;|
  !gapprompt@gap>| !gapinput@Size(allds);|
  294
  !gapprompt@gap>| !gapinput@A:=KramerMesnerMat(Group(()),Combinations([1..21],2),allds,1,21);;|
  !gapprompt@gap>| !gapinput@PAGGlobalOptions.Silent:=true;;|
  !gapprompt@gap>| !gapinput@sol:=AsSet(SolveKramerMesner(A));;|
  !gapprompt@gap>| !gapinput@des:=List(sol,x->BaseBlocks(allds,x));;|
  !gapprompt@gap>| !gapinput@Size(des);|
  70
\end{Verbatim}
 Among these $70$ designs, $14$ are left developments, and $14$ are right developments. The remaining $42$ designs are not developments, but all of their blocks are difference sets. 
\begin{Verbatim}[commandchars=!@|,fontsize=\small,frame=single,label=Example]
  !gapprompt@gap>| !gapinput@dev1:=AsSet(List(allds,x->LeftDevelopment(g[1],x).blocks));;|
  !gapprompt@gap>| !gapinput@Size(dev1);|
  14
  !gapprompt@gap>| !gapinput@dev2:=AsSet(List(allds,x->RightDevelopment(g[1],x).blocks));;|
  !gapprompt@gap>| !gapinput@Size(dev2);|
  14
  !gapprompt@gap>| !gapinput@nondev:=Difference(des,Union(dev1,dev2));;|
  !gapprompt@gap>| !gapinput@Size(nondev);|
  42
\end{Verbatim}
 Now we apply the group cube construction to these $42$ designs. The obtained cubes are equivalent. 
\begin{Verbatim}[commandchars=!@|,fontsize=\small,frame=single,label=Example]
  !gapprompt@gap>| !gapinput@cc:=List(nondev,x->GroupCube(g[1],x,3));;|
  !gapprompt@gap>| !gapinput@Size(CubeFilter(cc));|
  1
\end{Verbatim}
 The function \texttt{CubeFilter} (\ref{CubeFilter}) eliminates equivalent copies from a list of cubes. Our new cube is not
equivalent with the two $(21,5,1)$ difference cubes. 
\begin{Verbatim}[commandchars=!@|,fontsize=\small,frame=single,label=Example]
  !gapprompt@gap>| !gapinput@c3:=cc[1];;|
  !gapprompt@gap>| !gapinput@CubeTest(c3);|
  [ [ 21, 5, 1 ] ]
  !gapprompt@gap>| !gapinput@Size(CubeFilter([c1,c2,c3]));|
  3
  !gapprompt@gap>| !gapinput@Size(CubeAut(c3));|
  441
\end{Verbatim}
 However, the three cubes have the same slice invariant; see{\nobreakspace}\cite{KPT23} for the definition. 
\begin{Verbatim}[commandchars=!@|,fontsize=\small,frame=single,label=Example]
  !gapprompt@gap>| !gapinput@List([c1,c2,c3],SliceInvariant);|
  [ [ [ [ [ 120960, 21 ] ], 3 ] ], [ [ [ [ 120960, 21 ] ], 3 ] ], 
    [ [ [ [ 120960, 21 ] ], 3 ] ] ]
\end{Verbatim}
 Cubes with slice invariants different from any difference cube can be
constructed for parameters of the form $(4^m,2^{m-1}(2^m-1),2^{m-1}(2^{m-1}-1))$, $m\ge 2$. 
\begin{Verbatim}[commandchars=!@|,fontsize=\small,frame=single,label=Example]
  !gapprompt@gap>| !gapinput@m:=2;; n:=3;;|
  !gapprompt@gap>| !gapinput@cl:=List([1,2,3],i->GroupCube(SDPSeriesGroup(m),SDPSeriesDesign(m,i),n));;|
  !gapprompt@gap>| !gapinput@List(cl,CubeTest);|
  [ [ [ 16, 6, 2 ] ], [ [ 16, 6, 2 ] ], [ [ 16, 6, 2 ] ] ]
  !gapprompt@gap>| !gapinput@List(cl,SliceInvariant);|
  [ [ [ [ [ 11520, 16 ] ], 3 ] ], 
    [ [ [ [ 768, 16 ] ], 2 ], [ [ [ 11520, 16 ] ], 1 ] ], 
    [ [ [ [ 384, 16 ] ], 2 ], [ [ [ 11520, 16 ] ], 1 ] ] ]
\end{Verbatim}
 The first cube in the list \texttt{cl} is a difference cube. The other two cubes are not, because they have
non-isomorphic slices in different directions. This construction works for all $m\ge 2$ and dimensions $n\ge 3$, but it takes a lot of time and memory for bigger values of $m$ and{\nobreakspace}$n$. We classified all 3-dimensional group cubes of $(16,6,2)$ designs; they are available at \href{https://web.math.pmf.unizg.hr/~krcko/results/cubes.html} {\texttt{https://web.math.pmf.unizg.hr/\texttt{\symbol{126}}krcko/results/cubes.html}}. A list of 1423 non-group cubes of $(16,6,2)$ designs is also provided. 

 The package \textsf{DifSets} contains precomputed lists of difference sets up to equivalence. They are
loaded by the function \texttt{LoadDifferenceSets} (\textbf{DifSets: LoadDifferenceSets}). We can use them to compute all difference cubes up to equivalence. 
\begin{Verbatim}[commandchars=!@|,fontsize=\small,frame=single,label=Example]
  !gapprompt@gap>| !gapinput@v:=27;|
  27
  !gapprompt@gap>| !gapinput@l1:=Concatenation(List([1..NrSmallGroups(v)],|
  !gapprompt@>| !gapinput@i->List(LoadDifferenceSets(v,i),x->[i,x])));|
  [ [ 4, [ 1, 2, 3, 4, 5, 6, 9, 12, 16, 19, 20, 23, 26 ] ], 
    [ 4, [ 1, 2, 3, 4, 5, 7, 8, 9, 13, 15, 18, 19, 23 ] ], 
    [ 5, [ 1, 2, 3, 4, 5, 6, 7, 8, 9, 15, 23, 25, 27 ] ] ]
\end{Verbatim}
 The list \texttt{l1} now contains all inequivalent difference sets in groups of order 27. The first
entry is the group ID from the \textsf{GAP} library of small groups, followed by the difference set. 
\begin{Verbatim}[commandchars=!@|,fontsize=\small,frame=single,label=Example]
  !gapprompt@gap>| !gapinput@StructureDescription(SmallGroup(27,4));|
  "C9 : C3"
  !gapprompt@gap>| !gapinput@StructureDescription(SmallGroup(27,5));|
  "C3 x C3 x C3"
  !gapprompt@gap>| !gapinput@l2:=List(l1,x->DifferenceCube(SmallGroup(v,x[1]),x[2],3));;|
  !gapprompt@gap>| !gapinput@l3:=l1{CubeFilter(l2,rec(Positions:=true))};|
  [ [ 4, [ 1, 2, 3, 4, 5, 6, 9, 12, 16, 19, 20, 23, 26 ] ], 
    [ 5, [ 1, 2, 3, 4, 5, 6, 7, 8, 9, 15, 23, 25, 27 ] ] ]
\end{Verbatim}
 The list \texttt{l3} contains difference sets giving 3-cubes that are inequivalent (not paratopic).
Notice that the two cubes arising from difference sets in ${\ensuremath{\mathbb Z}}_9\rtimes {\ensuremath{\mathbb Z}}_3$ (group ID 4) are paratopic, but not isotopic: 
\begin{Verbatim}[commandchars=!@|,fontsize=\small,frame=single,label=Example]
  !gapprompt@gap>| !gapinput@l4:=l1{CubeFilter(l2,rec(Positions:=true,Isotopy:=true))};|
  [ [ 4, [ 1, 2, 3, 4, 5, 6, 9, 12, 16, 19, 20, 23, 26 ] ], 
    [ 4, [ 1, 2, 3, 4, 5, 7, 8, 9, 13, 15, 18, 19, 23 ] ], 
    [ 5, [ 1, 2, 3, 4, 5, 6, 7, 8, 9, 15, 23, 25, 27 ] ] ]
\end{Verbatim}
 

 We will now construct some non-difference group cubes in ${\ensuremath{\mathbb Z}}_9\rtimes {\ensuremath{\mathbb Z}}_3$. Here is an efficient way to get all difference sets, including equivalent
ones. 
\begin{Verbatim}[commandchars=!@|,fontsize=\small,frame=single,label=Example]
  !gapprompt@gap>| !gapinput@g:=SmallGroup(v,4);|
  <pc group of size 27 with 3 generators>
  !gapprompt@gap>| !gapinput@ge:=ExtendedPermRepresentation(g);|
  <permutation group with 7 generators>
  !gapprompt@gap>| !gapinput@ds:=LoadDifferenceSets(v,4);|
  [ [ 1, 2, 3, 4, 5, 6, 9, 12, 16, 19, 20, 23, 26 ], 
    [ 1, 2, 3, 4, 5, 7, 8, 9, 13, 15, 18, 19, 23 ] ]
  !gapprompt@gap>| !gapinput@allds:=Union(List(ds,x->Orbit(ge,x,OnSets)));;|
  !gapprompt@gap>| !gapinput@Size(allds);|
  972
\end{Verbatim}
 For parameters $(21,5,1)$ we could search for all designs with difference sets as blocks. This would
take too much time for $(27,13,6)$, so we prescribe an automorphism group of order{\nobreakspace}3. 
\begin{Verbatim}[commandchars=!@|,fontsize=\small,frame=single,label=Example]
  !gapprompt@gap>| !gapinput@sub:=AllSubgroupsConjugation(ge);;|
  !gapprompt@gap>| !gapinput@h:=sub[4];|
  Group([ (1,10,4)(2,15,7)(3,17,9)(5,20,12)(6,22,14)(8,23,16)
    (11,25,19)(13,26,21)(18,27,24) ])
  !gapprompt@gap>| !gapinput@alldsorb:=List(Orbits(h,allds,OnSets),Representative);;|
  !gapprompt@gap>| !gapinput@Size(alldsorb);|
  324
  !gapprompt@gap>| !gapinput@pairsorb:=List(Orbits(h,Combinations([1..27],2),OnSets),Representative);;|
  !gapprompt@gap>| !gapinput@Size(pairsorb);|
  117
  !gapprompt@gap>| !gapinput@A:=KramerMesnerMat(h,pairsorb,alldsorb,6,27);;|
  !gapprompt@gap>| !gapinput@sol:=AsSet(SolveKramerMesner(A));;|
  !gapprompt@gap>| !gapinput@des:=List(sol,x->BlockDesign(27,BaseBlocks(alldsorb,x),h).blocks);;|
  !gapprompt@gap>| !gapinput@Size(des);|
  288
\end{Verbatim}
 We get 288 designs with difference sets as blocks. Let us remove the ones
which are developments of their blocks. 
\begin{Verbatim}[commandchars=!@|,fontsize=\small,frame=single,label=Example]
  !gapprompt@gap>| !gapinput@dev1:=AsSet(List(allds,x->LeftDevelopment(g,x).blocks));;|
  !gapprompt@gap>| !gapinput@dev2:=AsSet(List(allds,x->RightDevelopment(g,x).blocks));;|
  !gapprompt@gap>| !gapinput@nondev:=List(Difference(des,Union(dev1,dev2)),x->[4,x]);;|
  !gapprompt@gap>| !gapinput@Size(nondev);|
  216
\end{Verbatim}
 Next, we remove the ones leading to equivalent 3-cubes. 
\begin{Verbatim}[commandchars=!@|,fontsize=\small,frame=single,label=Example]
  !gapprompt@gap>| !gapinput@cc:=List(nondev,x->GroupCube(SmallGroup(v,x[1]),x[2],3));;|
  !gapprompt@gap>| !gapinput@l5:=nondev{CubeFilter(cc,rec(Positions:=true))};|
  [ [ 4, 
        [ [ 1, 2, 3, 4, 5, 6, 9, 12, 16, 19, 20, 23, 26 ], 
            [ 1, 2, 3, 4, 5, 7, 10, 13, 14, 19, 21, 22, 24 ], 
            [ 1, 2, 3, 7, 11, 12, 13, 15, 20, 23, 24, 25, 27 ], 
            [ 1, 2, 4, 6, 10, 11, 13, 14, 15, 17, 18, 20, 26 ], 
            [ 1, 2, 4, 8, 9, 12, 13, 16, 17, 18, 22, 24, 27 ], 
            [ 1, 2, 9, 10, 11, 14, 16, 17, 19, 21, 23, 25, 27 ], 
            [ 1, 3, 4, 7, 8, 11, 17, 18, 19, 22, 23, 25, 26 ], 
            [ 1, 3, 5, 8, 9, 10, 14, 15, 18, 23, 24, 26, 27 ], 
            [ 1, 3, 5, 8, 10, 11, 12, 15, 16, 17, 20, 21, 22 ], 
            [ 1, 4, 6, 7, 9, 10, 12, 15, 21, 22, 25, 26, 27 ], 
            [ 1, 5, 6, 7, 9, 11, 14, 16, 18, 20, 22, 24, 25 ], 
            [ 1, 5, 6, 8, 13, 17, 19, 20, 21, 24, 25, 26, 27 ], 
            [ 1, 6, 7, 8, 12, 13, 14, 15, 16, 18, 19, 21, 23 ], 
            [ 2, 3, 5, 6, 9, 13, 15, 17, 18, 21, 22, 23, 25 ], 
            [ 2, 3, 6, 7, 8, 9, 11, 12, 14, 17, 21, 24, 26 ], 
            [ 2, 3, 6, 8, 10, 12, 14, 18, 19, 20, 22, 25, 27 ], 
            [ 2, 4, 5, 7, 8, 9, 11, 15, 18, 19, 20, 21, 27 ], 
            [ 2, 4, 8, 14, 15, 16, 20, 21, 22, 23, 24, 25, 26 ], 
            [ 2, 5, 6, 7, 8, 10, 11, 13, 16, 22, 23, 26, 27 ], 
            [ 2, 5, 7, 10, 12, 15, 16, 17, 18, 19, 24, 25, 26 ], 
            [ 3, 4, 5, 11, 12, 13, 14, 16, 18, 21, 25, 26, 27 ], 
            [ 3, 4, 6, 7, 10, 16, 17, 18, 20, 21, 23, 24, 27 ], 
            [ 3, 4, 6, 8, 9, 10, 11, 13, 15, 16, 19, 24, 25 ], 
            [ 3, 7, 9, 13, 14, 15, 16, 17, 19, 20, 22, 26, 27 ], 
            [ 4, 5, 6, 11, 12, 14, 15, 17, 19, 22, 23, 24, 27 ], 
            [ 4, 5, 7, 8, 9, 10, 12, 13, 14, 17, 20, 23, 25 ], 
            [ 9, 10, 11, 12, 13, 18, 19, 20, 21, 22, 23, 24, 26 ] ] ], 
    [ 4, 
        [ [ 1, 2, 3, 4, 5, 6, 9, 12, 16, 19, 20, 23, 26 ], 
            [ 1, 2, 3, 5, 7, 11, 14, 15, 18, 20, 23, 24, 25 ], 
            [ 1, 2, 3, 7, 9, 13, 14, 17, 19, 20, 21, 22, 27 ], 
            [ 1, 2, 4, 6, 7, 8, 10, 11, 13, 18, 20, 22, 26 ], 
            [ 1, 2, 4, 10, 12, 14, 15, 21, 22, 23, 25, 26, 27 ], 
            [ 1, 2, 5, 8, 12, 13, 17, 18, 19, 21, 24, 25, 26 ], 
            [ 1, 3, 4, 6, 8, 11, 13, 15, 17, 19, 23, 25, 27 ], 
            [ 1, 3, 5, 8, 10, 11, 12, 15, 16, 17, 20, 21, 22 ], 
            [ 1, 3, 6, 7, 8, 9, 10, 12, 18, 21, 23, 24, 27 ], 
            [ 1, 4, 5, 6, 7, 10, 13, 14, 15, 16, 19, 21, 24 ], 
            [ 1, 4, 8, 9, 14, 15, 16, 17, 18, 20, 24, 26, 27 ], 
            [ 1, 5, 6, 9, 11, 12, 13, 14, 16, 18, 22, 25, 27 ], 
            [ 1, 7, 9, 10, 11, 16, 17, 19, 22, 23, 24, 25, 26 ], 
            [ 2, 3, 4, 5, 10, 13, 16, 17, 18, 22, 23, 24, 27 ], 
            [ 2, 3, 4, 8, 9, 10, 11, 14, 16, 18, 19, 21, 25 ], 
            [ 2, 3, 6, 9, 10, 11, 12, 13, 14, 15, 17, 24, 26 ], 
            [ 2, 4, 5, 7, 8, 9, 11, 12, 15, 19, 22, 24, 27 ], 
            [ 2, 5, 6, 7, 8, 11, 14, 16, 17, 21, 23, 26, 27 ], 
            [ 2, 6, 7, 10, 12, 15, 16, 17, 18, 19, 20, 25, 27 ], 
            [ 2, 6, 8, 9, 13, 15, 16, 20, 21, 22, 23, 24, 25 ], 
            [ 3, 4, 5, 6, 7, 9, 15, 17, 18, 21, 22, 25, 26 ], 
            [ 3, 4, 7, 11, 12, 13, 16, 20, 21, 24, 25, 26, 27 ], 
            [ 3, 5, 6, 8, 10, 14, 19, 20, 22, 24, 25, 26, 27 ], 
            [ 3, 7, 8, 12, 13, 14, 15, 16, 18, 19, 22, 23, 26 ], 
            [ 4, 5, 7, 8, 9, 10, 12, 13, 14, 17, 20, 23, 25 ], 
            [ 4, 6, 11, 12, 14, 17, 18, 19, 20, 21, 22, 23, 24 ], 
            [ 5, 9, 10, 11, 13, 15, 18, 19, 20, 21, 23, 26, 27 ] ] ] ]
\end{Verbatim}
 We have constructed two $(27,13,6)$ designs with blocks being difference sets in ${\ensuremath{\mathbb Z}}_9\rtimes {\ensuremath{\mathbb Z}}_3$, which are not their developments. Here are the slice invariants of the
difference and non-difference group 3-cubes constructed so far. 
\begin{Verbatim}[commandchars=!@|,fontsize=\small,frame=single,label=Example]
  !gapprompt@gap>| !gapinput@dc:=List(l3,x->DifferenceCube(SmallGroup(v,x[1]),x[2],3));;|
  !gapprompt@gap>| !gapinput@gc:=List(l5,x->GroupCube(SmallGroup(v,x[1]),x[2],3));;|
  !gapprompt@gap>| !gapinput@List(dc,SliceInvariant);|
  [ [ [ [ [ 1053, 27 ] ], 3 ] ], [ [ [ [ 1053, 27 ] ], 3 ] ] ]
  !gapprompt@gap>| !gapinput@List(gc,SliceInvariant);|
  [ [ [ [ [ 27, 27 ] ], 2 ], [ [ [ 1053, 27 ] ], 1 ] ], 
    [ [ [ [ 27, 27 ] ], 2 ], [ [ [ 1053, 27 ] ], 1 ] ] ]
\end{Verbatim}
 More examples of difference and non-difference group cubes are available on
our web page: 

 \href{https://web.math.pmf.unizg.hr/~krcko/results/cubes.html} {\texttt{https://web.math.pmf.unizg.hr/\texttt{\symbol{126}}krcko/results/cubes.html}} }

 }

  
\chapter{\textcolor{Chapter }{The PAG Functions}}\label{The PAG Functions}
\logpage{[ 2, 0, 0 ]}
\hyperdef{L}{X7A34C81F843FE798}{}
{
  The following functions are available in the PAG package. 
\section{\textcolor{Chapter }{Working With Permutation Groups}}\label{Working With Permutation Groups}
\logpage{[ 2, 1, 0 ]}
\hyperdef{L}{X85FF48B18482CA86}{}
{
  

\subsection{\textcolor{Chapter }{CyclicPerm}}
\logpage{[ 2, 1, 1 ]}\nobreak
\hyperdef{L}{X79BF29A07905A90C}{}
{\noindent\textcolor{FuncColor}{$\triangleright$\enspace\texttt{CyclicPerm({\mdseries\slshape n})\index{CyclicPerm@\texttt{CyclicPerm}}
\label{CyclicPerm}
}\hfill{\scriptsize (function)}}\\


 Returns the cyclic permutation (1,...,\mbox{\texttt{\mdseries\slshape n}}). }

 

\subsection{\textcolor{Chapter }{ToGroup}}
\logpage{[ 2, 1, 2 ]}\nobreak
\hyperdef{L}{X7E1860E983EA72FC}{}
{\noindent\textcolor{FuncColor}{$\triangleright$\enspace\texttt{ToGroup({\mdseries\slshape G, f})\index{ToGroup@\texttt{ToGroup}}
\label{ToGroup}
}\hfill{\scriptsize (function)}}\\


 Apply function \mbox{\texttt{\mdseries\slshape f}} to each generator of the group \mbox{\texttt{\mdseries\slshape G}}. }

 

\subsection{\textcolor{Chapter }{MovePerm}}
\logpage{[ 2, 1, 3 ]}\nobreak
\hyperdef{L}{X80BEBE7B7A66BA4E}{}
{\noindent\textcolor{FuncColor}{$\triangleright$\enspace\texttt{MovePerm({\mdseries\slshape p, from, to})\index{MovePerm@\texttt{MovePerm}}
\label{MovePerm}
}\hfill{\scriptsize (function)}}\\


 Moves permutation \mbox{\texttt{\mdseries\slshape p}} acting on the set \mbox{\texttt{\mdseries\slshape from}} to a permutation acting on the set \mbox{\texttt{\mdseries\slshape to}}. The arguments \mbox{\texttt{\mdseries\slshape from}} and \mbox{\texttt{\mdseries\slshape to}} should be lists of integers of the same size. Alternatively, if instead of \mbox{\texttt{\mdseries\slshape from}} and \mbox{\texttt{\mdseries\slshape to}} just one integer argument \mbox{\texttt{\mdseries\slshape by}} is given, the permutation is moved from \texttt{MovedPoints(}\mbox{\texttt{\mdseries\slshape p}}\texttt{)} to \texttt{MovedPoints(}\mbox{\texttt{\mdseries\slshape p}}\texttt{)+}\mbox{\texttt{\mdseries\slshape by}}; see \texttt{MovedPoints} (\textbf{Reference: MovedPoints for a permutation}). }

 

\subsection{\textcolor{Chapter }{MoveGroup}}
\logpage{[ 2, 1, 4 ]}\nobreak
\hyperdef{L}{X81284A59843F7A7B}{}
{\noindent\textcolor{FuncColor}{$\triangleright$\enspace\texttt{MoveGroup({\mdseries\slshape G, from, to})\index{MoveGroup@\texttt{MoveGroup}}
\label{MoveGroup}
}\hfill{\scriptsize (function)}}\\


 Apply \texttt{MovePerm} (\ref{MovePerm}) to each generator of the group \mbox{\texttt{\mdseries\slshape G}}. }

 

\subsection{\textcolor{Chapter }{MultiPerm}}
\logpage{[ 2, 1, 5 ]}\nobreak
\hyperdef{L}{X7C03293E80B0CA72}{}
{\noindent\textcolor{FuncColor}{$\triangleright$\enspace\texttt{MultiPerm({\mdseries\slshape p, set, m})\index{MultiPerm@\texttt{MultiPerm}}
\label{MultiPerm}
}\hfill{\scriptsize (function)}}\\


 Repeat the action of a permutation \mbox{\texttt{\mdseries\slshape m}} times. The new permutation acts on \mbox{\texttt{\mdseries\slshape m}} disjoint copies of \mbox{\texttt{\mdseries\slshape set}}. }

 

\subsection{\textcolor{Chapter }{MultiGroup}}
\logpage{[ 2, 1, 6 ]}\nobreak
\hyperdef{L}{X7904788286E59E2E}{}
{\noindent\textcolor{FuncColor}{$\triangleright$\enspace\texttt{MultiGroup({\mdseries\slshape G, set, m})\index{MultiGroup@\texttt{MultiGroup}}
\label{MultiGroup}
}\hfill{\scriptsize (function)}}\\


 Apply \texttt{MultiPerm} (\ref{MultiPerm}) to each generator of the group \mbox{\texttt{\mdseries\slshape G}}. }

 

\subsection{\textcolor{Chapter }{RestrictedGroup}}
\logpage{[ 2, 1, 7 ]}\nobreak
\hyperdef{L}{X790683938254CDC1}{}
{\noindent\textcolor{FuncColor}{$\triangleright$\enspace\texttt{RestrictedGroup({\mdseries\slshape G, set})\index{RestrictedGroup@\texttt{RestrictedGroup}}
\label{RestrictedGroup}
}\hfill{\scriptsize (function)}}\\


 Apply \texttt{RestrictedPerm} (\textbf{Reference: RestrictedPerm}) to each generator of the group \mbox{\texttt{\mdseries\slshape G}}. }

 

\subsection{\textcolor{Chapter }{PrimitiveGroupsOfDegree}}
\logpage{[ 2, 1, 8 ]}\nobreak
\hyperdef{L}{X7C6BC92378E6B4BC}{}
{\noindent\textcolor{FuncColor}{$\triangleright$\enspace\texttt{PrimitiveGroupsOfDegree({\mdseries\slshape v})\index{PrimitiveGroupsOfDegree@\texttt{PrimitiveGroupsOfDegree}}
\label{PrimitiveGroupsOfDegree}
}\hfill{\scriptsize (function)}}\\


 Returns a list of all primitive permutation groups on \mbox{\texttt{\mdseries\slshape v}} points. }

 

\subsection{\textcolor{Chapter }{TransitiveGroupsOfDegree}}
\logpage{[ 2, 1, 9 ]}\nobreak
\hyperdef{L}{X8589DDAD7DB0FCA8}{}
{\noindent\textcolor{FuncColor}{$\triangleright$\enspace\texttt{TransitiveGroupsOfDegree({\mdseries\slshape v})\index{TransitiveGroupsOfDegree@\texttt{TransitiveGroupsOfDegree}}
\label{TransitiveGroupsOfDegree}
}\hfill{\scriptsize (function)}}\\


 Returns a list of all transitive permutation groups on \mbox{\texttt{\mdseries\slshape v}} points. }

 

\subsection{\textcolor{Chapter }{AllSubgroupsConjugation}}
\logpage{[ 2, 1, 10 ]}\nobreak
\hyperdef{L}{X81A70E877DF8AA34}{}
{\noindent\textcolor{FuncColor}{$\triangleright$\enspace\texttt{AllSubgroupsConjugation({\mdseries\slshape G})\index{AllSubgroupsConjugation@\texttt{AllSubgroupsConjugation}}
\label{AllSubgroupsConjugation}
}\hfill{\scriptsize (function)}}\\


 Returns a list of all subgroups of \mbox{\texttt{\mdseries\slshape G}} up to conjugation. }

 

\subsection{\textcolor{Chapter }{PermRepresentationRight}}
\logpage{[ 2, 1, 11 ]}\nobreak
\hyperdef{L}{X780C40F4873006FB}{}
{\noindent\textcolor{FuncColor}{$\triangleright$\enspace\texttt{PermRepresentationRight({\mdseries\slshape G})\index{PermRepresentationRight@\texttt{PermRepresentationRight}}
\label{PermRepresentationRight}
}\hfill{\scriptsize (function)}}\\


 Returns the regular permutation representation of a group \mbox{\texttt{\mdseries\slshape G}} by right multiplication. }

 

\subsection{\textcolor{Chapter }{PermRepresentationLeft}}
\logpage{[ 2, 1, 12 ]}\nobreak
\hyperdef{L}{X815AE3827C9CA969}{}
{\noindent\textcolor{FuncColor}{$\triangleright$\enspace\texttt{PermRepresentationLeft({\mdseries\slshape G})\index{PermRepresentationLeft@\texttt{PermRepresentationLeft}}
\label{PermRepresentationLeft}
}\hfill{\scriptsize (function)}}\\


 Returns the regular permutation representation of a group \mbox{\texttt{\mdseries\slshape G}} by left multiplication. }

 

\subsection{\textcolor{Chapter }{ExtendedPermRepresentation}}
\logpage{[ 2, 1, 13 ]}\nobreak
\hyperdef{L}{X7BD56AE78333F3A2}{}
{\noindent\textcolor{FuncColor}{$\triangleright$\enspace\texttt{ExtendedPermRepresentation({\mdseries\slshape G})\index{ExtendedPermRepresentation@\texttt{ExtendedPermRepresentation}}
\label{ExtendedPermRepresentation}
}\hfill{\scriptsize (function)}}\\


 Returns the extended permutation representation of a group \mbox{\texttt{\mdseries\slshape G}} including right multiplication, left multiplication, and group automorphisms. }

 }

 
\section{\textcolor{Chapter }{Generating Orbits}}\label{Generating Ortbits}
\logpage{[ 2, 2, 0 ]}
\hyperdef{L}{X7CA95AB080E5135D}{}
{
  

\subsection{\textcolor{Chapter }{SubsetOrbitRep}}
\logpage{[ 2, 2, 1 ]}\nobreak
\hyperdef{L}{X7C1945A784AC5F72}{}
{\noindent\textcolor{FuncColor}{$\triangleright$\enspace\texttt{SubsetOrbitRep({\mdseries\slshape G, v, k[, opt]})\index{SubsetOrbitRep@\texttt{SubsetOrbitRep}}
\label{SubsetOrbitRep}
}\hfill{\scriptsize (function)}}\\


 Computes orbit representatives of \mbox{\texttt{\mdseries\slshape k}}-subsets of [1..\mbox{\texttt{\mdseries\slshape v}}] under the action of the permutation group \mbox{\texttt{\mdseries\slshape G}}. The basic algorithm is described in \cite{KVK21}. The algorithm for short orbits is described in \cite{KV16}. The last argument is a record \mbox{\texttt{\mdseries\slshape opt}} for options. The possible components of \mbox{\texttt{\mdseries\slshape opt}} are: 
\begin{itemize}
\item \mbox{\texttt{\mdseries\slshape SizeLE}}:=\mbox{\texttt{\mdseries\slshape n}} If defined, only representatives of orbits of size less or equal to \mbox{\texttt{\mdseries\slshape n}} are computed.
\item \mbox{\texttt{\mdseries\slshape IntesectionNumbers}}:=\mbox{\texttt{\mdseries\slshape lin}} If defined, only representatives of good orbits are returned. These are orbits
with intersection numbers in the list of integers \mbox{\texttt{\mdseries\slshape lin}}.
\end{itemize}
 }

 

\subsection{\textcolor{Chapter }{SubsetOrbitRepShort1}}
\logpage{[ 2, 2, 2 ]}\nobreak
\hyperdef{L}{X7A6453417DBD5FC1}{}
{\noindent\textcolor{FuncColor}{$\triangleright$\enspace\texttt{SubsetOrbitRepShort1({\mdseries\slshape G, v, k, size})\index{SubsetOrbitRepShort1@\texttt{SubsetOrbitRepShort1}}
\label{SubsetOrbitRepShort1}
}\hfill{\scriptsize (function)}}\\


 Computes \mbox{\texttt{\mdseries\slshape G}}-orbit representatives of \mbox{\texttt{\mdseries\slshape k}}-subsets of [1..\mbox{\texttt{\mdseries\slshape v}}] of size less or equal \mbox{\texttt{\mdseries\slshape size}}. Here, \mbox{\texttt{\mdseries\slshape size}} is an integer smaller than the order of the group \mbox{\texttt{\mdseries\slshape G}}. The algorithm is described in \cite{KV16}. }

 

\subsection{\textcolor{Chapter }{SubsetOrbitRepIN}}
\logpage{[ 2, 2, 3 ]}\nobreak
\hyperdef{L}{X81601D33790912F8}{}
{\noindent\textcolor{FuncColor}{$\triangleright$\enspace\texttt{SubsetOrbitRepIN({\mdseries\slshape G, v, k, lin[, opt]})\index{SubsetOrbitRepIN@\texttt{SubsetOrbitRepIN}}
\label{SubsetOrbitRepIN}
}\hfill{\scriptsize (function)}}\\


 Computes orbit representatives of \mbox{\texttt{\mdseries\slshape k}}-subsets of [1..\mbox{\texttt{\mdseries\slshape v}}] under the action of the permutation group \mbox{\texttt{\mdseries\slshape G}} with intersection numbers in the list \mbox{\texttt{\mdseries\slshape lin}}. Parts of the search tree with partial subsets intersecting in more than the
largest number in \mbox{\texttt{\mdseries\slshape lin}} are skipped. Short orbits are computed separately. The algorithm is described
in \cite{KVK21}. The last (optional) argument \mbox{\texttt{\mdseries\slshape opt}} is a record for options. The possible components are: 
\begin{itemize}
\item \mbox{\texttt{\mdseries\slshape Verbose}}:=\texttt{true}/\texttt{false} Print comments reporting the progress of the calculation.
\item \mbox{\texttt{\mdseries\slshape FilteringLevel}}:=\mbox{\texttt{\mdseries\slshape n}} Apply filrering of the search tree up to subsets of size \mbox{\texttt{\mdseries\slshape n}}. By default, \mbox{\texttt{\mdseries\slshape n}}=\mbox{\texttt{\mdseries\slshape k}}.
\end{itemize}
 }

 

\subsection{\textcolor{Chapter }{IsGoodSubsetOrbit}}
\logpage{[ 2, 2, 4 ]}\nobreak
\hyperdef{L}{X821B07AF82565DCE}{}
{\noindent\textcolor{FuncColor}{$\triangleright$\enspace\texttt{IsGoodSubsetOrbit({\mdseries\slshape G, rep, lin})\index{IsGoodSubsetOrbit@\texttt{IsGoodSubsetOrbit}}
\label{IsGoodSubsetOrbit}
}\hfill{\scriptsize (function)}}\\


 Check if the subset orbit generated by the permutation group \mbox{\texttt{\mdseries\slshape G}} and the representative \mbox{\texttt{\mdseries\slshape rep}} is a good orbit with respect to the list of intersection numbers \mbox{\texttt{\mdseries\slshape lin}}. This means that the intersection size of any pair of sets from the orbit is
an integer in \mbox{\texttt{\mdseries\slshape lin}}. }

 

\subsection{\textcolor{Chapter }{SmallLambdaFilter}}
\logpage{[ 2, 2, 5 ]}\nobreak
\hyperdef{L}{X84B02FCA8105895E}{}
{\noindent\textcolor{FuncColor}{$\triangleright$\enspace\texttt{SmallLambdaFilter({\mdseries\slshape G, tsub, ksub, lambda})\index{SmallLambdaFilter@\texttt{SmallLambdaFilter}}
\label{SmallLambdaFilter}
}\hfill{\scriptsize (function)}}\\


 Remove $k$-subset representatives from \mbox{\texttt{\mdseries\slshape ksub}} such that the corresponding \mbox{\texttt{\mdseries\slshape G}}-orbit covers some of the $t$-subset representatives from \mbox{\texttt{\mdseries\slshape tsub}} more than \mbox{\texttt{\mdseries\slshape lambda}} times. }

 

\subsection{\textcolor{Chapter }{OrbitFilter1}}
\logpage{[ 2, 2, 6 ]}\nobreak
\hyperdef{L}{X7DDEA540864CCB29}{}
{\noindent\textcolor{FuncColor}{$\triangleright$\enspace\texttt{OrbitFilter1({\mdseries\slshape G, obj, action})\index{OrbitFilter1@\texttt{OrbitFilter1}}
\label{OrbitFilter1}
}\hfill{\scriptsize (function)}}\\


 Takes a list of objects \mbox{\texttt{\mdseries\slshape obj}} and returns one representative from each orbit of the group \mbox{\texttt{\mdseries\slshape G}} acting by \mbox{\texttt{\mdseries\slshape action}}. The result is a sublist of \mbox{\texttt{\mdseries\slshape obj}}. The algorithm uses the \textsf{GAP} function \texttt{Orbit} (\textbf{Reference: Orbit}). }

 

\subsection{\textcolor{Chapter }{OrbitFilter2}}
\logpage{[ 2, 2, 7 ]}\nobreak
\hyperdef{L}{X844E305B81352606}{}
{\noindent\textcolor{FuncColor}{$\triangleright$\enspace\texttt{OrbitFilter2({\mdseries\slshape G, obj, action})\index{OrbitFilter2@\texttt{OrbitFilter2}}
\label{OrbitFilter2}
}\hfill{\scriptsize (function)}}\\


 Takes a list of objects \mbox{\texttt{\mdseries\slshape obj}} and returns one representative from each orbit of the group \mbox{\texttt{\mdseries\slshape G}} acting by \mbox{\texttt{\mdseries\slshape action}}. Canonical representatives are returned, so the result is not a sublist of \mbox{\texttt{\mdseries\slshape obj}}. The algorithm uses the \texttt{CanonicalImage} (\textbf{images: CanonicalImage}) function from the package \textsf{Images}. }

 }

 
\section{\textcolor{Chapter }{Constructing Objects}}\label{Constructing Objects}
\logpage{[ 2, 3, 0 ]}
\hyperdef{L}{X81E58D178665D555}{}
{
  

\subsection{\textcolor{Chapter }{KramerMesnerSearch}}
\logpage{[ 2, 3, 1 ]}\nobreak
\hyperdef{L}{X853E396684BE1D17}{}
{\noindent\textcolor{FuncColor}{$\triangleright$\enspace\texttt{KramerMesnerSearch({\mdseries\slshape t, v, k, lambda, G[, opt]})\index{KramerMesnerSearch@\texttt{KramerMesnerSearch}}
\label{KramerMesnerSearch}
}\hfill{\scriptsize (function)}}\\


 Performs a search for \mbox{\texttt{\mdseries\slshape t}}-(\mbox{\texttt{\mdseries\slshape v}},\mbox{\texttt{\mdseries\slshape k}},\mbox{\texttt{\mdseries\slshape lambda}}) designs with presrcribed automorphism group \mbox{\texttt{\mdseries\slshape G}} by the Kramer-Mesner method. A record with options can be supplied. By
default, designs are returned in the \textsf{Design} package format  (\textbf{DESIGN: Design}) and isomorph-rejection is performed by calling \texttt{BlockDesignFilter} (\ref{BlockDesignFilter}). It can be turned off by setting \mbox{\texttt{\mdseries\slshape opt.NonIsomorphic}}:=\texttt{false}. By setting \mbox{\texttt{\mdseries\slshape opt.BaseBlocks}}:=\texttt{true}, base blocks are returned instead of designs. This automatically turns off
isomorph-rejection. Other available options are: 
\begin{itemize}
\item \mbox{\texttt{\mdseries\slshape SmallLambda}}:=\texttt{true}/\texttt{false}. Perform the ``small lambda filter'', i.e. remove \mbox{\texttt{\mdseries\slshape k}}-orbits covering some of the \mbox{\texttt{\mdseries\slshape t}}-orbits more than \mbox{\texttt{\mdseries\slshape lambda}} times. By default, this is done if \mbox{\texttt{\mdseries\slshape lambda}}{\textless}=3.
\item \mbox{\texttt{\mdseries\slshape IntersectionNumbers}}:=\mbox{\texttt{\mdseries\slshape lin}}/\texttt{false}. Search for designs with block intersection nubers in the list of integers \mbox{\texttt{\mdseries\slshape lin}} (e.g. quasi-symmetric designs).
\end{itemize}
 }

 

\subsection{\textcolor{Chapter }{KramerMesnerMat}}
\logpage{[ 2, 3, 2 ]}\nobreak
\hyperdef{L}{X831AFBCE828421DC}{}
{\noindent\textcolor{FuncColor}{$\triangleright$\enspace\texttt{KramerMesnerMat({\mdseries\slshape G, tsub, ksub[, lambda][, b]})\index{KramerMesnerMat@\texttt{KramerMesnerMat}}
\label{KramerMesnerMat}
}\hfill{\scriptsize (function)}}\\


 Returns the Kramer-Mesner matrix for a permutation group \mbox{\texttt{\mdseries\slshape G}}. The rows are labelled by $t$-subset orbits represented by \mbox{\texttt{\mdseries\slshape tsub}}, and the columns by $k$-subset orbits represented by \mbox{\texttt{\mdseries\slshape ksub}}. A column of constants \mbox{\texttt{\mdseries\slshape lambda}} is added if the optional argument \mbox{\texttt{\mdseries\slshape lambda}} is given. Another row is added if the optional argument \mbox{\texttt{\mdseries\slshape b}} is given, repesenting the constraint that sizes of the chosen $k$-subset orbits must sum up to the number of blocks \mbox{\texttt{\mdseries\slshape b}}. }

 

\subsection{\textcolor{Chapter }{CompatibilityMat}}
\logpage{[ 2, 3, 3 ]}\nobreak
\hyperdef{L}{X79E1DA2E802C2EDA}{}
{\noindent\textcolor{FuncColor}{$\triangleright$\enspace\texttt{CompatibilityMat({\mdseries\slshape G, ksub, lin})\index{CompatibilityMat@\texttt{CompatibilityMat}}
\label{CompatibilityMat}
}\hfill{\scriptsize (function)}}\\


 Returns the compatibility matrix of the $k$-subset representatives \mbox{\texttt{\mdseries\slshape ksub}} with respect to the group \mbox{\texttt{\mdseries\slshape G}} and list of intersection numbers \mbox{\texttt{\mdseries\slshape lin}}. Entries are $1$ if intersection sizes of subsets in the corresponding \mbox{\texttt{\mdseries\slshape G}}-orbits are all integers in \mbox{\texttt{\mdseries\slshape lin}}, and $0$ otherwise. }

 

\subsection{\textcolor{Chapter }{SolveKramerMesner}}
\logpage{[ 2, 3, 4 ]}\nobreak
\hyperdef{L}{X7A9D5D9A7FB92630}{}
{\noindent\textcolor{FuncColor}{$\triangleright$\enspace\texttt{SolveKramerMesner({\mdseries\slshape mat[, cm][, opt]})\index{SolveKramerMesner@\texttt{SolveKramerMesner}}
\label{SolveKramerMesner}
}\hfill{\scriptsize (function)}}\\


 Solve a system of linear equations determined by the matrix \mbox{\texttt{\mdseries\slshape mat}} over $\{0,1\}$. By default, A.Wassermann's LLL solver \texttt{solvediophant} \cite{AW98} is used. If the second argument is a compatibility matrix \mbox{\texttt{\mdseries\slshape cm}}, the backtracking program \texttt{solvecm} from the papers \cite{KNP11} and \cite{KV16} is used. The solver can also be chosen explicitly in the record \mbox{\texttt{\mdseries\slshape opt}}. Possible components are: 
\begin{itemize}
\item \mbox{\texttt{\mdseries\slshape Solver}}:=\texttt{"solvediophant"} If defined, \texttt{solvediophant} is used.
\item \mbox{\texttt{\mdseries\slshape Solver}}:=\texttt{"solvecm"} If defined, \texttt{solvecm} is used.
\end{itemize}
 }

 

\subsection{\textcolor{Chapter }{BaseBlocks}}
\logpage{[ 2, 3, 5 ]}\nobreak
\hyperdef{L}{X80F64ADB7EC13F7E}{}
{\noindent\textcolor{FuncColor}{$\triangleright$\enspace\texttt{BaseBlocks({\mdseries\slshape ksub, sol})\index{BaseBlocks@\texttt{BaseBlocks}}
\label{BaseBlocks}
}\hfill{\scriptsize (function)}}\\


 Returns base blocks of design(s) from solution(s) \mbox{\texttt{\mdseries\slshape sol}} by picking them from $k$-subset orbit representatives \mbox{\texttt{\mdseries\slshape ksub}}. }

 

\subsection{\textcolor{Chapter }{ExpandMatRHS}}
\logpage{[ 2, 3, 6 ]}\nobreak
\hyperdef{L}{X86582C2D7F941DCA}{}
{\noindent\textcolor{FuncColor}{$\triangleright$\enspace\texttt{ExpandMatRHS({\mdseries\slshape mat, lambda})\index{ExpandMatRHS@\texttt{ExpandMatRHS}}
\label{ExpandMatRHS}
}\hfill{\scriptsize (function)}}\\


 Add a column of \mbox{\texttt{\mdseries\slshape lambda}}'s to the right of the matrix \mbox{\texttt{\mdseries\slshape mat}}. }

 

\subsection{\textcolor{Chapter }{RightDevelopment}}
\logpage{[ 2, 3, 7 ]}\nobreak
\hyperdef{L}{X83B4006C82C7DCC7}{}
{\noindent\textcolor{FuncColor}{$\triangleright$\enspace\texttt{RightDevelopment({\mdseries\slshape G, ds})\index{RightDevelopment@\texttt{RightDevelopment}}
\label{RightDevelopment}
}\hfill{\scriptsize (function)}}\\


 Returns a block design that is the development of the difference set \mbox{\texttt{\mdseries\slshape ds}} by right multiplication in the group \mbox{\texttt{\mdseries\slshape G}}. }

 

\subsection{\textcolor{Chapter }{LeftDevelopment}}
\logpage{[ 2, 3, 8 ]}\nobreak
\hyperdef{L}{X81C414057EC904AC}{}
{\noindent\textcolor{FuncColor}{$\triangleright$\enspace\texttt{LeftDevelopment({\mdseries\slshape G, ds})\index{LeftDevelopment@\texttt{LeftDevelopment}}
\label{LeftDevelopment}
}\hfill{\scriptsize (function)}}\\


 Returns a block design that is the development of the difference set \mbox{\texttt{\mdseries\slshape ds}} by left multiplication in the group \mbox{\texttt{\mdseries\slshape G}}. }

 }

 
\section{\textcolor{Chapter }{Inspecting Objects and Other Functions}}\label{Inspecting Objects and Other Functions}
\logpage{[ 2, 4, 0 ]}
\hyperdef{L}{X7A5A75EE7D024A4A}{}
{
  

\subsection{\textcolor{Chapter }{BlockDesignAut}}
\logpage{[ 2, 4, 1 ]}\nobreak
\hyperdef{L}{X805C0B9F831607DC}{}
{\noindent\textcolor{FuncColor}{$\triangleright$\enspace\texttt{BlockDesignAut({\mdseries\slshape d[, opt]})\index{BlockDesignAut@\texttt{BlockDesignAut}}
\label{BlockDesignAut}
}\hfill{\scriptsize (function)}}\\


 Computes the full automorphism group of a block design \mbox{\texttt{\mdseries\slshape d}}. Uses \texttt{nauty/Traces 2.8} by B.D.McKay and A.Piperno \cite{MP14}. This is an alternative for the \texttt{AutGroupBlockDesign} function from the \textsf{Design} package  (\textbf{DESIGN: Automorphism groups and isomorphism testing for block designs}). The optional argument \mbox{\texttt{\mdseries\slshape opt}} is a record for options. Possible components of \mbox{\texttt{\mdseries\slshape opt}} are: 
\begin{itemize}
\item \mbox{\texttt{\mdseries\slshape Traces}}:=\texttt{true}/\texttt{false} Use \texttt{Traces}. This is the default.
\item \mbox{\texttt{\mdseries\slshape SparseNauty}}:=\texttt{true}/\texttt{false} Use \texttt{nauty} for sparse graphs.
\item \mbox{\texttt{\mdseries\slshape DenseNauty}}:=\texttt{true}/\texttt{false} Use \texttt{nauty} for dense graphs. This is usually the slowest program, but it allows vertex
invariants. Vertex invariants are ignored by the other programs.
\item \mbox{\texttt{\mdseries\slshape BlockAction}}:=\texttt{true}/\texttt{false} If set to \texttt{true}, the action of the automorphisms on blocks is also given. In this case
automorphisms are permutations of degree $v+b$. By default, only the action on points is given, i.e. automorphisms are
permutations of degree $v$.
\item \mbox{\texttt{\mdseries\slshape Dual}}:=\texttt{true}/\texttt{false} If set to \texttt{true}, dual automorphisms (correlations) are also included. They will appear only
for self-dual symmetric designs (with the same number of points and blocks).
The default is \texttt{false}.
\item \mbox{\texttt{\mdseries\slshape PointClasses}}:=\mbox{\texttt{\mdseries\slshape s}} Color the points into classes of size \mbox{\texttt{\mdseries\slshape s}} that cannot be mapped onto each other. By default all points are in the same
class.
\item \mbox{\texttt{\mdseries\slshape VertexInvariant}}:=\mbox{\texttt{\mdseries\slshape n}} Use vertex invariant number \mbox{\texttt{\mdseries\slshape n}}. The numbering is the same as in \texttt{dreadnaut}, e.g. \mbox{\texttt{\mdseries\slshape n}}=1: \texttt{twopaths}, \mbox{\texttt{\mdseries\slshape n}}=2: \texttt{adjtriang}, etc. The default is \texttt{twopaths}. Vertex invariants only work with dense \texttt{nauty}. They are ignored by sparse \texttt{nauty} and \texttt{Traces}.
\item \mbox{\texttt{\mdseries\slshape Mininvarlevel}}:=\mbox{\texttt{\mdseries\slshape n}} Set \texttt{mininvarlevel} to \mbox{\texttt{\mdseries\slshape n}}. The default is \mbox{\texttt{\mdseries\slshape n}}=0.
\item \mbox{\texttt{\mdseries\slshape Maxinvarlevel}}:=\mbox{\texttt{\mdseries\slshape n}} Set \texttt{maxinvarlevel} to \mbox{\texttt{\mdseries\slshape n}}. The default is \mbox{\texttt{\mdseries\slshape n}}=2.
\item \mbox{\texttt{\mdseries\slshape Invararg}}:=\mbox{\texttt{\mdseries\slshape n}} Set \texttt{invararg} to \mbox{\texttt{\mdseries\slshape n}}. The default is \mbox{\texttt{\mdseries\slshape n}}=0.
\end{itemize}
 }

 

\subsection{\textcolor{Chapter }{BlockDesignFilter}}
\logpage{[ 2, 4, 2 ]}\nobreak
\hyperdef{L}{X7C629C8280899316}{}
{\noindent\textcolor{FuncColor}{$\triangleright$\enspace\texttt{BlockDesignFilter({\mdseries\slshape dl[, opt]})\index{BlockDesignFilter@\texttt{BlockDesignFilter}}
\label{BlockDesignFilter}
}\hfill{\scriptsize (function)}}\\


 Eliminates isomorphic copies from a list of block designs \mbox{\texttt{\mdseries\slshape dl}}. Uses \texttt{nauty/Traces 2.8} by B.D.McKay and A.Piperno \cite{MP14}. This is an alternative for the \texttt{BlockDesignIsomorphismClassRepresentatives} function from the \textsf{Design} package  (\textbf{DESIGN: Automorphism groups and isomorphism testing for block designs}). The optional argument \mbox{\texttt{\mdseries\slshape opt}} is a record for options. Possible components of \mbox{\texttt{\mdseries\slshape opt}} are: 
\begin{itemize}
\item \mbox{\texttt{\mdseries\slshape Traces}}:=\texttt{true}/\texttt{false} Use \texttt{Traces}. This is the default.
\item \mbox{\texttt{\mdseries\slshape SparseNauty}}:=\texttt{true}/\texttt{false} Use \texttt{nauty} for sparse graphs.
\item \mbox{\texttt{\mdseries\slshape PointClasses}}:=\mbox{\texttt{\mdseries\slshape s}} Color the points into classes of size \mbox{\texttt{\mdseries\slshape s}} that cannot be mapped onto each other. By default all points are in the same
class.
\item \mbox{\texttt{\mdseries\slshape Positions}}:=\texttt{true}/\texttt{false} Return positions of nonisomorphic designs instead of the designs themselves.
\end{itemize}
 }

 

\subsection{\textcolor{Chapter }{IntersectionNumbers}}
\logpage{[ 2, 4, 3 ]}\nobreak
\hyperdef{L}{X864491248518B7CD}{}
{\noindent\textcolor{FuncColor}{$\triangleright$\enspace\texttt{IntersectionNumbers({\mdseries\slshape d[, opt]})\index{IntersectionNumbers@\texttt{IntersectionNumbers}}
\label{IntersectionNumbers}
}\hfill{\scriptsize (function)}}\\


 Returns the list of intersection numbers of the block design \mbox{\texttt{\mdseries\slshape d}}. The optional argument \mbox{\texttt{\mdseries\slshape opt}} is a record for options. Possible components of \mbox{\texttt{\mdseries\slshape opt}} are: 
\begin{itemize}
\item \mbox{\texttt{\mdseries\slshape Frequencies}}:=\texttt{true}/\texttt{false} If set to \texttt{true}, frequencies of the intersection numbers are also returned.
\end{itemize}
 }

 

\subsection{\textcolor{Chapter }{BlockScheme}}
\logpage{[ 2, 4, 4 ]}\nobreak
\hyperdef{L}{X824E5FCE8742EB70}{}
{\noindent\textcolor{FuncColor}{$\triangleright$\enspace\texttt{BlockScheme({\mdseries\slshape d})\index{BlockScheme@\texttt{BlockScheme}}
\label{BlockScheme}
}\hfill{\scriptsize (function)}}\\


 Returns the block intersection scheme of a schematic block design \mbox{\texttt{\mdseries\slshape d}}. If \mbox{\texttt{\mdseries\slshape d}} is not schematic, returns \texttt{fail}. Uses the package \textsf{AssociationSchemes}. }

 

\subsection{\textcolor{Chapter }{TDesignB}}
\logpage{[ 2, 4, 5 ]}\nobreak
\hyperdef{L}{X858A707E81380706}{}
{\noindent\textcolor{FuncColor}{$\triangleright$\enspace\texttt{TDesignB({\mdseries\slshape t, v, k, lambda})\index{TDesignB@\texttt{TDesignB}}
\label{TDesignB}
}\hfill{\scriptsize (function)}}\\


 The number of blocks of a \mbox{\texttt{\mdseries\slshape t}}-(\mbox{\texttt{\mdseries\slshape v}},\mbox{\texttt{\mdseries\slshape k}},\mbox{\texttt{\mdseries\slshape lambda}}) design. }

 

\subsection{\textcolor{Chapter }{IversonBracket}}
\logpage{[ 2, 4, 6 ]}\nobreak
\hyperdef{L}{X8386E2DC7CDC30D2}{}
{\noindent\textcolor{FuncColor}{$\triangleright$\enspace\texttt{IversonBracket({\mdseries\slshape P})\index{IversonBracket@\texttt{IversonBracket}}
\label{IversonBracket}
}\hfill{\scriptsize (function)}}\\


 Returns 1 if \mbox{\texttt{\mdseries\slshape P}} is true, and 0 otherwise. }

 

\subsection{\textcolor{Chapter }{SymmetricDifference}}
\logpage{[ 2, 4, 7 ]}\nobreak
\hyperdef{L}{X86F40A1A87BF4E96}{}
{\noindent\textcolor{FuncColor}{$\triangleright$\enspace\texttt{SymmetricDifference({\mdseries\slshape X, Y})\index{SymmetricDifference@\texttt{SymmetricDifference}}
\label{SymmetricDifference}
}\hfill{\scriptsize (function)}}\\


 Returns the symmetric difference of two sets \mbox{\texttt{\mdseries\slshape X}} and \mbox{\texttt{\mdseries\slshape Y}}. }

 }

 
\section{\textcolor{Chapter }{Latin Squares}}\label{Latin Squares}
\logpage{[ 2, 5, 0 ]}
\hyperdef{L}{X7889CA477AC29A8C}{}
{
  

\subsection{\textcolor{Chapter }{ReadMOLS}}
\logpage{[ 2, 5, 1 ]}\nobreak
\hyperdef{L}{X7A81988C8176227D}{}
{\noindent\textcolor{FuncColor}{$\triangleright$\enspace\texttt{ReadMOLS({\mdseries\slshape filename})\index{ReadMOLS@\texttt{ReadMOLS}}
\label{ReadMOLS}
}\hfill{\scriptsize (function)}}\\


 Read a list of MOLS sets from a file. The file starts with the number of rows $m$, columns $n$, and the size of the sets $s$, followed by the matrix entries. Integers in the file are separated by
whitespaces. }

 

\subsection{\textcolor{Chapter }{WriteMOLS}}
\logpage{[ 2, 5, 2 ]}\nobreak
\hyperdef{L}{X7E57B98986C1F138}{}
{\noindent\textcolor{FuncColor}{$\triangleright$\enspace\texttt{WriteMOLS({\mdseries\slshape filename, list})\index{WriteMOLS@\texttt{WriteMOLS}}
\label{WriteMOLS}
}\hfill{\scriptsize (function)}}\\


 Write a list of MOLS sets to a file. The number of rows $m$, columns $n$, and the size of the sets $s$ is written first, followed by the matrix entries. Integers are separated by
whitespaces. }

 

\subsection{\textcolor{Chapter }{CayleyTableOfGroup}}
\logpage{[ 2, 5, 3 ]}\nobreak
\hyperdef{L}{X87AF98127C4CD44B}{}
{\noindent\textcolor{FuncColor}{$\triangleright$\enspace\texttt{CayleyTableOfGroup({\mdseries\slshape G})\index{CayleyTableOfGroup@\texttt{CayleyTableOfGroup}}
\label{CayleyTableOfGroup}
}\hfill{\scriptsize (function)}}\\


 Returns a Cayley table of the group \mbox{\texttt{\mdseries\slshape G}}. The elements are integers $1,\ldots,$\texttt{Order(}\mbox{\texttt{\mdseries\slshape G}}\texttt{)}. }

 

\subsection{\textcolor{Chapter }{FieldToMOLS}}
\logpage{[ 2, 5, 4 ]}\nobreak
\hyperdef{L}{X7C72080B8777A3E8}{}
{\noindent\textcolor{FuncColor}{$\triangleright$\enspace\texttt{FieldToMOLS({\mdseries\slshape F})\index{FieldToMOLS@\texttt{FieldToMOLS}}
\label{FieldToMOLS}
}\hfill{\scriptsize (function)}}\\


 Construct a complete set of MOLS from the finite field \mbox{\texttt{\mdseries\slshape F}}. A similar function is \texttt{MOLS} (\textbf{GUAVA: MOLS}) from the package \textsf{Guava}. }

 

\subsection{\textcolor{Chapter }{MOLSToOrthogonalArray}}
\logpage{[ 2, 5, 5 ]}\nobreak
\hyperdef{L}{X7F1F5D9A833584F7}{}
{\noindent\textcolor{FuncColor}{$\triangleright$\enspace\texttt{MOLSToOrthogonalArray({\mdseries\slshape ls})\index{MOLSToOrthogonalArray@\texttt{MOLSToOrthogonalArray}}
\label{MOLSToOrthogonalArray}
}\hfill{\scriptsize (function)}}\\


 Transforms the set of MOLS \mbox{\texttt{\mdseries\slshape ls}} to an equivalent orthogonal array. }

 

\subsection{\textcolor{Chapter }{OrthogonalArrayToMOLS}}
\logpage{[ 2, 5, 6 ]}\nobreak
\hyperdef{L}{X7E7B537D83F87531}{}
{\noindent\textcolor{FuncColor}{$\triangleright$\enspace\texttt{OrthogonalArrayToMOLS({\mdseries\slshape oa})\index{OrthogonalArrayToMOLS@\texttt{OrthogonalArrayToMOLS}}
\label{OrthogonalArrayToMOLS}
}\hfill{\scriptsize (function)}}\\


 Transforms the orthogonal array \mbox{\texttt{\mdseries\slshape oa}} to an equivalent set of MOLS. }

 

\subsection{\textcolor{Chapter }{MOLSToTransversalDesign}}
\logpage{[ 2, 5, 7 ]}\nobreak
\hyperdef{L}{X84CA3E8478200A8A}{}
{\noindent\textcolor{FuncColor}{$\triangleright$\enspace\texttt{MOLSToTransversalDesign({\mdseries\slshape ls})\index{MOLSToTransversalDesign@\texttt{MOLSToTransversalDesign}}
\label{MOLSToTransversalDesign}
}\hfill{\scriptsize (function)}}\\


 Transforms the set of MOLS \mbox{\texttt{\mdseries\slshape ls}} to an equivalent transversal design. }

 

\subsection{\textcolor{Chapter }{TransversalDesignToMOLS}}
\logpage{[ 2, 5, 8 ]}\nobreak
\hyperdef{L}{X84C2E30C8354947C}{}
{\noindent\textcolor{FuncColor}{$\triangleright$\enspace\texttt{TransversalDesignToMOLS({\mdseries\slshape td})\index{TransversalDesignToMOLS@\texttt{TransversalDesignToMOLS}}
\label{TransversalDesignToMOLS}
}\hfill{\scriptsize (function)}}\\


 Transforms the transversal design \mbox{\texttt{\mdseries\slshape td}} to an equivalent set of MOLS. }

 

\subsection{\textcolor{Chapter }{MOLSAut}}
\logpage{[ 2, 5, 9 ]}\nobreak
\hyperdef{L}{X7D15F68E7A55A95F}{}
{\noindent\textcolor{FuncColor}{$\triangleright$\enspace\texttt{MOLSAut({\mdseries\slshape ls[, opt]})\index{MOLSAut@\texttt{MOLSAut}}
\label{MOLSAut}
}\hfill{\scriptsize (function)}}\\


 Computes the full auto(para)topy group of a set of MOLS \mbox{\texttt{\mdseries\slshape ls}}. Uses \texttt{nauty/Traces 2.8} by B.D.McKay and A.Piperno \cite{MP14}. The optional argument \mbox{\texttt{\mdseries\slshape opt}} is a record for options. Possible components are: 
\begin{itemize}
\item \mbox{\texttt{\mdseries\slshape Isotopy}}:=\texttt{true}/\texttt{false} Compute the full autotopy group of \mbox{\texttt{\mdseries\slshape ls}}. This is the default.
\item \mbox{\texttt{\mdseries\slshape Paratopy}}:=\texttt{true}/\texttt{false} Compute the full autoparatopy group of \mbox{\texttt{\mdseries\slshape ls}}.
\end{itemize}
 Any other components will be forwarded to the \texttt{BlockDesignAut} (\ref{BlockDesignAut}) function; see its documentation. }

 

\subsection{\textcolor{Chapter }{MOLSFilter}}
\logpage{[ 2, 5, 10 ]}\nobreak
\hyperdef{L}{X826116DB845FC8E8}{}
{\noindent\textcolor{FuncColor}{$\triangleright$\enspace\texttt{MOLSFilter({\mdseries\slshape ls[, opt]})\index{MOLSFilter@\texttt{MOLSFilter}}
\label{MOLSFilter}
}\hfill{\scriptsize (function)}}\\


 Eliminates isotopic/paratopic copies from a list of MOLS sets \mbox{\texttt{\mdseries\slshape ls}}. Uses \texttt{nauty/Traces 2.8} by B.D.McKay and A.Piperno \cite{MP14}. The optional argument \mbox{\texttt{\mdseries\slshape opt}} is a record for options. Possible components are: 
\begin{itemize}
\item \mbox{\texttt{\mdseries\slshape Paratopy}}:=\texttt{true}/\texttt{false} Eliminate paratopic MOLS sets. This is the default.
\item \mbox{\texttt{\mdseries\slshape Isotopy}}:=\texttt{true}/\texttt{false} Eliminate isotopic MOLS sets.
\end{itemize}
 Any other components will be forwarded to the \texttt{BlockDesignFilter} (\ref{BlockDesignFilter}) function; see its documentation. }

 

\subsection{\textcolor{Chapter }{IsAutotopyGroup}}
\logpage{[ 2, 5, 11 ]}\nobreak
\hyperdef{L}{X85D053247FA1924B}{}
{\noindent\textcolor{FuncColor}{$\triangleright$\enspace\texttt{IsAutotopyGroup({\mdseries\slshape n, s, G})\index{IsAutotopyGroup@\texttt{IsAutotopyGroup}}
\label{IsAutotopyGroup}
}\hfill{\scriptsize (function)}}\\


 Check if \mbox{\texttt{\mdseries\slshape G}} is an autotopy group for transversal designs with \mbox{\texttt{\mdseries\slshape s}}$+2$ point classes of order \mbox{\texttt{\mdseries\slshape n}}. }

 

\subsection{\textcolor{Chapter }{MOLSSubsetOrbitRep}}
\logpage{[ 2, 5, 12 ]}\nobreak
\hyperdef{L}{X782EAB457FC77498}{}
{\noindent\textcolor{FuncColor}{$\triangleright$\enspace\texttt{MOLSSubsetOrbitRep({\mdseries\slshape n, s, G})\index{MOLSSubsetOrbitRep@\texttt{MOLSSubsetOrbitRep}}
\label{MOLSSubsetOrbitRep}
}\hfill{\scriptsize (function)}}\\


 Computes representatives of pairs and transversals of the \mbox{\texttt{\mdseries\slshape s}}$+2$ point classes for the construction of MOLS of order \mbox{\texttt{\mdseries\slshape n}} with prescribed autotopy group \mbox{\texttt{\mdseries\slshape G}}. A list containing pair representatives in the first component and
transversal representatives in the second component is returned. }

 

\subsection{\textcolor{Chapter }{KramerMesnerMOLS}}
\logpage{[ 2, 5, 13 ]}\nobreak
\hyperdef{L}{X87FD64DB7C954BA1}{}
{\noindent\textcolor{FuncColor}{$\triangleright$\enspace\texttt{KramerMesnerMOLS({\mdseries\slshape n, s, G[, opt]})\index{KramerMesnerMOLS@\texttt{KramerMesnerMOLS}}
\label{KramerMesnerMOLS}
}\hfill{\scriptsize (function)}}\\


 If the function \texttt{IsAutotopyGroup} (\ref{IsAutotopyGroup})(\mbox{\texttt{\mdseries\slshape G}}) returns \texttt{true} for the group \mbox{\texttt{\mdseries\slshape G}}, call \texttt{KramerMesnerMOLSAutotopy} (\ref{KramerMesnerMOLSAutotopy}); otherwise call \texttt{KramerMesnerMOLSAutoparatopy} (\ref{KramerMesnerMOLSAutoparatopy}). }

 

\subsection{\textcolor{Chapter }{KramerMesnerMOLSAutotopy}}
\logpage{[ 2, 5, 14 ]}\nobreak
\hyperdef{L}{X7861AB617AC13AF5}{}
{\noindent\textcolor{FuncColor}{$\triangleright$\enspace\texttt{KramerMesnerMOLSAutotopy({\mdseries\slshape n, s, G[, opt]})\index{KramerMesnerMOLSAutotopy@\texttt{KramerMesnerMOLSAutotopy}}
\label{KramerMesnerMOLSAutotopy}
}\hfill{\scriptsize (function)}}\\


 Search for MOLS sets of order \mbox{\texttt{\mdseries\slshape n}} and size \mbox{\texttt{\mdseries\slshape s}} with prescribed autotopy group \mbox{\texttt{\mdseries\slshape G}}. By default, A.Wassermann's LLL solver \texttt{solvediophant} is used for \mbox{\texttt{\mdseries\slshape s}}$=1$, and the backtracking solver \texttt{solvecm} is used for \mbox{\texttt{\mdseries\slshape s}}$>1$. This can be changed by setting options in the record \mbox{\texttt{\mdseries\slshape opt}}. Available options are: 
\begin{itemize}
\item \mbox{\texttt{\mdseries\slshape Solver}}:=\texttt{"solvediophant"} Use \texttt{solvediophant}.
\item \mbox{\texttt{\mdseries\slshape Solver}}:=\texttt{"solvecm"} Use \texttt{solvecm}.
\item \mbox{\texttt{\mdseries\slshape Paratopy}}:=\texttt{true}/\texttt{false} Eliminate paratopic solutions. This is the default.
\item \mbox{\texttt{\mdseries\slshape Isotopy}}:=\texttt{true}/\texttt{false} Eliminate isotopic solutions. All solutions are returned if either option is
set to \texttt{false}.
\end{itemize}
 }

 

\subsection{\textcolor{Chapter }{KramerMesnerMOLSAutoparatopy}}
\logpage{[ 2, 5, 15 ]}\nobreak
\hyperdef{L}{X87AFFFA97D37406B}{}
{\noindent\textcolor{FuncColor}{$\triangleright$\enspace\texttt{KramerMesnerMOLSAutoparatopy({\mdseries\slshape n, s, G[, opt]})\index{KramerMesnerMOLSAutoparatopy@\texttt{KramerMesnerMOLSAutoparatopy}}
\label{KramerMesnerMOLSAutoparatopy}
}\hfill{\scriptsize (function)}}\\


 Search for MOLS sets of order \mbox{\texttt{\mdseries\slshape n}} and size \mbox{\texttt{\mdseries\slshape s}} with prescribed autoparatopy group \mbox{\texttt{\mdseries\slshape G}}. By default, A.Wassermann's LLL solver \texttt{solvediophant} is used for \mbox{\texttt{\mdseries\slshape s}}$=1$, and the backtracking solver \texttt{solvecm} is used for \mbox{\texttt{\mdseries\slshape s}}$>1$. This can be changed by setting options in the record \mbox{\texttt{\mdseries\slshape opt}}. Available options are: 
\begin{itemize}
\item \mbox{\texttt{\mdseries\slshape Solver}}:=\texttt{"solvediophant"} Use \texttt{solvediophant}.
\item \mbox{\texttt{\mdseries\slshape Solver}}:=\texttt{"solvecm"} Use \texttt{solvecm}.
\item \mbox{\texttt{\mdseries\slshape Paratopy}}:=\texttt{true}/\texttt{false} Eliminate paratopic solutions. This is the default.
\item \mbox{\texttt{\mdseries\slshape Isotopy}}:=\texttt{true}/\texttt{false} Eliminate isotopic solutions. All solutions are returned if either option is
set to \texttt{false}.
\end{itemize}
 }

 }

 
\section{\textcolor{Chapter }{Cubes of Symmetric Designs}}\label{Cubes of Symmetric Designs}
\logpage{[ 2, 6, 0 ]}
\hyperdef{L}{X79F7C22885C658DE}{}
{
  

\subsection{\textcolor{Chapter }{DifferenceCube}}
\logpage{[ 2, 6, 1 ]}\nobreak
\hyperdef{L}{X796E426D877E220A}{}
{\noindent\textcolor{FuncColor}{$\triangleright$\enspace\texttt{DifferenceCube({\mdseries\slshape G, ds, n})\index{DifferenceCube@\texttt{DifferenceCube}}
\label{DifferenceCube}
}\hfill{\scriptsize (function)}}\\


 Returns the \mbox{\texttt{\mdseries\slshape n}}-dimenional difference cube constructed from a difference set \mbox{\texttt{\mdseries\slshape ds}} in the group \mbox{\texttt{\mdseries\slshape G}}. }

 

\subsection{\textcolor{Chapter }{GroupCube}}
\logpage{[ 2, 6, 2 ]}\nobreak
\hyperdef{L}{X7D20EEDC8566D8BF}{}
{\noindent\textcolor{FuncColor}{$\triangleright$\enspace\texttt{GroupCube({\mdseries\slshape G, dds, n})\index{GroupCube@\texttt{GroupCube}}
\label{GroupCube}
}\hfill{\scriptsize (function)}}\\


 Returns the \mbox{\texttt{\mdseries\slshape n}}-dimenional group cube constructed from a symmetric design \mbox{\texttt{\mdseries\slshape dds}} such that the blocks are difference sets in the group \mbox{\texttt{\mdseries\slshape G}}. }

 

\subsection{\textcolor{Chapter }{CubeSlice}}
\logpage{[ 2, 6, 3 ]}\nobreak
\hyperdef{L}{X7F4C2D7579CA223C}{}
{\noindent\textcolor{FuncColor}{$\triangleright$\enspace\texttt{CubeSlice({\mdseries\slshape C, x, y, fixed})\index{CubeSlice@\texttt{CubeSlice}}
\label{CubeSlice}
}\hfill{\scriptsize (function)}}\\


 Returns a 2-dimensional slice of the incidence cube \mbox{\texttt{\mdseries\slshape C}} obtained by varying coordinates in positions \mbox{\texttt{\mdseries\slshape x}} and \mbox{\texttt{\mdseries\slshape y}}, and taking fixed values for the remaining coordinates given in a list \mbox{\texttt{\mdseries\slshape fixed}}. }

 

\subsection{\textcolor{Chapter }{CubeSlices}}
\logpage{[ 2, 6, 4 ]}\nobreak
\hyperdef{L}{X7D2F947285ECC89F}{}
{\noindent\textcolor{FuncColor}{$\triangleright$\enspace\texttt{CubeSlices({\mdseries\slshape C[, x, y][, fixed]})\index{CubeSlices@\texttt{CubeSlices}}
\label{CubeSlices}
}\hfill{\scriptsize (function)}}\\


 Returns 2-dimensional slices of the incidence cube \mbox{\texttt{\mdseries\slshape C}}. Optional arguments are the varying coordinates \mbox{\texttt{\mdseries\slshape x}} and \mbox{\texttt{\mdseries\slshape y}}, and values of the fixed coordinates in a list \mbox{\texttt{\mdseries\slshape fixed}}. If optional arguments are not given, all possibilities will be supplied. For
an $n$-dimensional cube \mbox{\texttt{\mdseries\slshape C}} of order $v$, the following calls will return: 
\begin{itemize}
\item CubeSlices( \mbox{\texttt{\mdseries\slshape C}}, \mbox{\texttt{\mdseries\slshape x}}, \mbox{\texttt{\mdseries\slshape y}} ) $\ldots v^{n-2}$ slices obtained by varying values of the fixed coordinates.
\item CubeSlices( \mbox{\texttt{\mdseries\slshape C}}, \mbox{\texttt{\mdseries\slshape fixed}} ) $\ldots {n\choose 2}$ slices obtained by varying the non-fixed coordinates $x < y$.
\item CubeSlices( \mbox{\texttt{\mdseries\slshape C}} ) $\ldots {n\choose 2}\cdot v^{n-2}$ slices obtained by varying both the non-fixed coordinates $x < y$ and values of the fixed coordinates.
\end{itemize}
 }

 

\subsection{\textcolor{Chapter }{CubeToOrthogonalArray}}
\logpage{[ 2, 6, 5 ]}\nobreak
\hyperdef{L}{X8214B28081B21BDE}{}
{\noindent\textcolor{FuncColor}{$\triangleright$\enspace\texttt{CubeToOrthogonalArray({\mdseries\slshape C})\index{CubeToOrthogonalArray@\texttt{CubeToOrthogonalArray}}
\label{CubeToOrthogonalArray}
}\hfill{\scriptsize (function)}}\\


 Transforms the incidence cube \mbox{\texttt{\mdseries\slshape C}} to an equivalent orthogonal array. }

 

\subsection{\textcolor{Chapter }{OrthogonalArrayToCube}}
\logpage{[ 2, 6, 6 ]}\nobreak
\hyperdef{L}{X87C8CFE97ECDC3DA}{}
{\noindent\textcolor{FuncColor}{$\triangleright$\enspace\texttt{OrthogonalArrayToCube({\mdseries\slshape oa})\index{OrthogonalArrayToCube@\texttt{OrthogonalArrayToCube}}
\label{OrthogonalArrayToCube}
}\hfill{\scriptsize (function)}}\\


 Transforms the orthogonal array \mbox{\texttt{\mdseries\slshape oa}} to an equivalent incidence cube. }

 

\subsection{\textcolor{Chapter }{CubeToTransversalDesign}}
\logpage{[ 2, 6, 7 ]}\nobreak
\hyperdef{L}{X8281E4697AA795A3}{}
{\noindent\textcolor{FuncColor}{$\triangleright$\enspace\texttt{CubeToTransversalDesign({\mdseries\slshape C})\index{CubeToTransversalDesign@\texttt{CubeToTransversalDesign}}
\label{CubeToTransversalDesign}
}\hfill{\scriptsize (function)}}\\


 Transforms the incidence cube \mbox{\texttt{\mdseries\slshape C}} to an equivalent transversal design. }

 

\subsection{\textcolor{Chapter }{TransversalDesignToCube}}
\logpage{[ 2, 6, 8 ]}\nobreak
\hyperdef{L}{X7D717F98781699EF}{}
{\noindent\textcolor{FuncColor}{$\triangleright$\enspace\texttt{TransversalDesignToCube({\mdseries\slshape td})\index{TransversalDesignToCube@\texttt{TransversalDesignToCube}}
\label{TransversalDesignToCube}
}\hfill{\scriptsize (function)}}\\


 Transforms the transversal design \mbox{\texttt{\mdseries\slshape td}} to an equivalent incidence cube. }

 

\subsection{\textcolor{Chapter }{LatinSquareToCube}}
\logpage{[ 2, 6, 9 ]}\nobreak
\hyperdef{L}{X8699117579A6173C}{}
{\noindent\textcolor{FuncColor}{$\triangleright$\enspace\texttt{LatinSquareToCube({\mdseries\slshape L})\index{LatinSquareToCube@\texttt{LatinSquareToCube}}
\label{LatinSquareToCube}
}\hfill{\scriptsize (function)}}\\


 Transforms the Latin square \mbox{\texttt{\mdseries\slshape L}} to an equivalent incidence cube. }

 

\subsection{\textcolor{Chapter }{CubeTest}}
\logpage{[ 2, 6, 10 ]}\nobreak
\hyperdef{L}{X7B79A7857DAD2FFE}{}
{\noindent\textcolor{FuncColor}{$\triangleright$\enspace\texttt{CubeTest({\mdseries\slshape C})\index{CubeTest@\texttt{CubeTest}}
\label{CubeTest}
}\hfill{\scriptsize (function)}}\\


 Test whether an incidence cube \mbox{\texttt{\mdseries\slshape C}} is a cube of symmetric designs. The result should be \texttt{[[v,k,lambda]]}. Anything else means that \mbox{\texttt{\mdseries\slshape C}} is not a $(v,k,lambda)$ cube. }

 

\subsection{\textcolor{Chapter }{SliceInvariant}}
\logpage{[ 2, 6, 11 ]}\nobreak
\hyperdef{L}{X781264CA816BE32E}{}
{\noindent\textcolor{FuncColor}{$\triangleright$\enspace\texttt{SliceInvariant({\mdseries\slshape C})\index{SliceInvariant@\texttt{SliceInvariant}}
\label{SliceInvariant}
}\hfill{\scriptsize (function)}}\\


 Computes a paratopy invariant of the cube \mbox{\texttt{\mdseries\slshape C}} based on automorphism group sizes of parallel slices. Cubes equivalent under
paratopy have the same invariant. }

 

\subsection{\textcolor{Chapter }{CubeAut}}
\logpage{[ 2, 6, 12 ]}\nobreak
\hyperdef{L}{X7E96F84678D23676}{}
{\noindent\textcolor{FuncColor}{$\triangleright$\enspace\texttt{CubeAut({\mdseries\slshape C[, opt]})\index{CubeAut@\texttt{CubeAut}}
\label{CubeAut}
}\hfill{\scriptsize (function)}}\\


 Computes the full auto(para)topy group of an incidence cube \mbox{\texttt{\mdseries\slshape C}}. Uses \texttt{nauty/Traces 2.8} by B.D.McKay and A.Piperno \cite{MP14}. The optional argument \mbox{\texttt{\mdseries\slshape opt}} is a record for options. Possible components are: 
\begin{itemize}
\item \mbox{\texttt{\mdseries\slshape Isotopy}}:=\texttt{true}/\texttt{false} Compute the full autotopy group of \mbox{\texttt{\mdseries\slshape C}}. This is the default.
\item \mbox{\texttt{\mdseries\slshape Paratopy}}:=\texttt{true}/\texttt{false} Compute the full autoparatopy group of \mbox{\texttt{\mdseries\slshape C}}.
\end{itemize}
 Any other components will be forwarded to the \texttt{BlockDesignAut} (\ref{BlockDesignAut}) function; see its documentation. }

 

\subsection{\textcolor{Chapter }{CubeFilter}}
\logpage{[ 2, 6, 13 ]}\nobreak
\hyperdef{L}{X810B22E286D857C1}{}
{\noindent\textcolor{FuncColor}{$\triangleright$\enspace\texttt{CubeFilter({\mdseries\slshape cl[, opt]})\index{CubeFilter@\texttt{CubeFilter}}
\label{CubeFilter}
}\hfill{\scriptsize (function)}}\\


 Eliminates equivalent copies from a list of incidence cubes \mbox{\texttt{\mdseries\slshape cl}}. Uses \texttt{nauty/Traces 2.8} by B.D.McKay and A.Piperno \cite{MP14}. The optional argument \mbox{\texttt{\mdseries\slshape opt}} is a record for options. Possible components are: 
\begin{itemize}
\item \mbox{\texttt{\mdseries\slshape Paratopy}}:=\texttt{true}/\texttt{false} Eliminate paratopic cubes. This is the default.
\item \mbox{\texttt{\mdseries\slshape Isotopy}}:=\texttt{true}/\texttt{false} Eliminate isotopic cubes.
\end{itemize}
 Any other components will be forwarded to the \texttt{BlockDesignFilter} (\ref{BlockDesignFilter}) function; see its documentation. }

 

\subsection{\textcolor{Chapter }{SDPSeriesGroup}}
\logpage{[ 2, 6, 14 ]}\nobreak
\hyperdef{L}{X853C796F7AE3EAB3}{}
{\noindent\textcolor{FuncColor}{$\triangleright$\enspace\texttt{SDPSeriesGroup({\mdseries\slshape m})\index{SDPSeriesGroup@\texttt{SDPSeriesGroup}}
\label{SDPSeriesGroup}
}\hfill{\scriptsize (function)}}\\


 Returns a group for the designs of \texttt{SDPSeriesDesign} (\ref{SDPSeriesDesign}). This is the elementary Abelian group of order $4^m$. }

 

\subsection{\textcolor{Chapter }{SDPSeriesDesign}}
\logpage{[ 2, 6, 15 ]}\nobreak
\hyperdef{L}{X840763A67E94AE26}{}
{\noindent\textcolor{FuncColor}{$\triangleright$\enspace\texttt{SDPSeriesDesign({\mdseries\slshape m, i})\index{SDPSeriesDesign@\texttt{SDPSeriesDesign}}
\label{SDPSeriesDesign}
}\hfill{\scriptsize (function)}}\\


 Returns a symmetric block design with parameters $(4^m,2^{m-1}(2^m-1),2^{m-1}(2^{m-1}-1))$. The argument \mbox{\texttt{\mdseries\slshape i}} must be 1, 2, or 3. If \mbox{\texttt{\mdseries\slshape i}}$=1$, the design is the symplectic design of Kantor \cite{WK75}. This design has the symmetric difference property (SDP). If \mbox{\texttt{\mdseries\slshape i}}$=2$ or \mbox{\texttt{\mdseries\slshape i}}$=3$, two other non-isomorphic designs with the same parameters are returned. They
are not SDP designs, but have the property that all their blocks are
difference sets in the group returned by \texttt{SDPSeriesGroup} (\ref{SDPSeriesGroup}). Developments of these blocks are isomorphic to the design for \mbox{\texttt{\mdseries\slshape i}}$=1$, so the two other designs are not developments of their blocks. }

 }

 
\section{\textcolor{Chapter }{Hadamard Matrices}}\label{Hadamard Matrices}
\logpage{[ 2, 7, 0 ]}
\hyperdef{L}{X7AD0B7D486954F10}{}
{
  

\subsection{\textcolor{Chapter }{HadamardMatAut}}
\logpage{[ 2, 7, 1 ]}\nobreak
\hyperdef{L}{X801DEB1E7D5668E9}{}
{\noindent\textcolor{FuncColor}{$\triangleright$\enspace\texttt{HadamardMatAut({\mdseries\slshape H[, opt]})\index{HadamardMatAut@\texttt{HadamardMatAut}}
\label{HadamardMatAut}
}\hfill{\scriptsize (function)}}\\


 Computes the full automorphism group of a Hadamard matrix \mbox{\texttt{\mdseries\slshape H}}. Represents the matrix by a colored graph (see \cite{BM79}) and uses \texttt{nauty/Traces 2.8} by B.D.McKay and A.Piperno \cite{MP14}. The optional argument \mbox{\texttt{\mdseries\slshape opt}} is a record for options. Possible components of \mbox{\texttt{\mdseries\slshape opt}} are: 
\begin{itemize}
\item \mbox{\texttt{\mdseries\slshape Dual}}:=\texttt{true}/\texttt{false} If set to \texttt{true}, dual automorphisms (transpositions) are also allowed. The default is \texttt{false}.
\end{itemize}
 }

 

\subsection{\textcolor{Chapter }{HadamardMatFilter}}
\logpage{[ 2, 7, 2 ]}\nobreak
\hyperdef{L}{X7E0014F67EC9FC23}{}
{\noindent\textcolor{FuncColor}{$\triangleright$\enspace\texttt{HadamardMatFilter({\mdseries\slshape hl[, opt]})\index{HadamardMatFilter@\texttt{HadamardMatFilter}}
\label{HadamardMatFilter}
}\hfill{\scriptsize (function)}}\\


 Eliminates equivalent copies from a list of Hadamard matrices \mbox{\texttt{\mdseries\slshape hl}}. Represents the matrices by colored graphs (see \cite{BM79}) and uses \texttt{nauty/Traces 2.8} by B.D.McKay and A.Piperno \cite{MP14}. The optional argument \mbox{\texttt{\mdseries\slshape opt}} is a record for options. Possible components of \mbox{\texttt{\mdseries\slshape opt}} are: 
\begin{itemize}
\item \mbox{\texttt{\mdseries\slshape Dual}}:=\texttt{true}/\texttt{false} If set to \texttt{true}, dual equivalence is allowed (i.e. the matrices can be transposed). The
default is \texttt{false}.
\item \mbox{\texttt{\mdseries\slshape Positions}}:=\texttt{true}/\texttt{false} Return positions of inequivalent Hadamard matrices instead of the matrices
themselves.
\end{itemize}
 }

 

\subsection{\textcolor{Chapter }{SDPSeriesHadamardMat}}
\logpage{[ 2, 7, 3 ]}\nobreak
\hyperdef{L}{X805C567178DC9FD8}{}
{\noindent\textcolor{FuncColor}{$\triangleright$\enspace\texttt{SDPSeriesHadamardMat({\mdseries\slshape m, i})\index{SDPSeriesHadamardMat@\texttt{SDPSeriesHadamardMat}}
\label{SDPSeriesHadamardMat}
}\hfill{\scriptsize (function)}}\\


 Returns a Hadamard matrix of order $4^m$ for the SDP series of designs. The argument \mbox{\texttt{\mdseries\slshape i}} must be 1, 2, or 3. See documentation for the \texttt{SDPSeriesDesign} (\ref{SDPSeriesDesign}) function. }

 }

 
\section{\textcolor{Chapter }{Global Options}}\label{Global Options}
\logpage{[ 2, 8, 0 ]}
\hyperdef{L}{X78E96F5385881B1B}{}
{
  

\subsection{\textcolor{Chapter }{PAGGlobalOptions}}
\logpage{[ 2, 8, 1 ]}\nobreak
\hyperdef{L}{X8626782884DAC536}{}
{\noindent\textcolor{FuncColor}{$\triangleright$\enspace\texttt{PAGGlobalOptions\index{PAGGlobalOptions@\texttt{PAGGlobalOptions}}
\label{PAGGlobalOptions}
}\hfill{\scriptsize (global variable)}}\\


 A record with global options for the PAG package. Components are: 
\begin{itemize}
\item \mbox{\texttt{\mdseries\slshape Silent}}:=\texttt{true}/\texttt{false} If set to \texttt{true}, functions such as SolveKramerMesner will not print comments reporting the
progress of the calculation.
\item \mbox{\texttt{\mdseries\slshape TempDir}}:=\texttt{directory object} Temporary directory used to communicate with external programs.
\end{itemize}
 }

 }

 }

    {\nobreakspace} \def\bibname{References\logpage{[ "Bib", 0, 0 ]}
\hyperdef{L}{X7A6F98FD85F02BFE}{}
}

\bibliographystyle{alpha}
\bibliography{bib.xml}

\addcontentsline{toc}{chapter}{References}

\def\indexname{Index\logpage{[ "Ind", 0, 0 ]}
\hyperdef{L}{X83A0356F839C696F}{}
}

\cleardoublepage
\phantomsection
\addcontentsline{toc}{chapter}{Index}


\printindex

\immediate\write\pagenrlog{["Ind", 0, 0], \arabic{page},}
\newpage
\immediate\write\pagenrlog{["End"], \arabic{page}];}
\immediate\closeout\pagenrlog
\end{document}
